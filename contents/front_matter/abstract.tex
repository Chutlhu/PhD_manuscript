\chapter*{Abstract}\addcontentsline{toc}{chapter}{Abstract}

Block ciphers form, without doubt, the backbone of today's encrypted communication and are thus justifiably the workhorses of cryptography.
While efficiency of modern designs improved ever since the development of the DES and AES, the case with the corresponding security arguments differs.
The thesis at hand aims at two main points, both in the direction of improving security analysis of block ciphers.

Part~I studies a new notion for the better understanding of a special type of cryptanalysis and proposes a new block cipher instance.
This instance comes with a tight bound on any differential, to the best of our knowledge the first such block cipher.

Part~II turns to automated methods in design and analysis of block ciphers.
Our main contribution here is an algorithm to propagate subspaces through encryption rounds, together with two applications: an algorithmic security argument against a new type of cryptanalysis and an idea towards the automation of key recovery attacks.
