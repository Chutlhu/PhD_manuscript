\chapter*{Abstract}\addcontentsline{toc}{chapter}{Abstract}

{   \small

    Audio scene analysis aims at retrieving useful information from microphone recordings.
    Examples of these problems are sound source separation and sound source localization, where we are interested in estimating a speaker's content and position.
    As humans, we perform these tasks without effort. However, for computers and robots, they are still open challenges.
    One of the main limitations is that most of the available technologies solve audio scene analysis problems considering either sound semantic or spatial information.

    \mynewline
    The central theme of this theses is acoustic echoes: the sound propagation elements bridging between semantic and spatial information of sound sources.
    In particular, as repetitions a source sound, their semantic contribution can be integrated to enhance the sound source.
    As originated by the interaction with the environment, their path can be backtracked and used to estimate the audio scene's geometry.
    Based on these observations, recent echo-aware audio signal processing methods have been proposed.
    However, two are the main questions that arise: how we estimated acoustic echoes, and how we use their knowledge?

    \mynewline
    This thesis work aims at improving the current state-of-the-art for indoor audio signal processing along these two axes.
    It also provides new methodologies and data to process acoustic echoes and surpass current approaches' limits.
    To this end, we present two approaches:
    A novel approach based on the  continuous dictionary framework which does not rely on parameter tuning or peak picking techniques;
    A deep learning model estimating the time difference of arrival of the first prominent echoes using physically-motivated regularizers.
    Furthermore, we present a fully annotated dataset specifically designed acoustic echo retrieval, echo-aware application for validating future echo-aware research.

    \mynewline
    The second part of this thesis regards extending existing methods for audio scene analysis in their echo-aware form.
    The Multichannel NMF framework for audio source separation, the SRP-PHAT localization methods, and the MVDR beamformer for speech enhancement
    are re-proposed in their echo-aware version. These applications show how a simple echo model can lead to a boost in performance.

    \mynewline
    Finally, we want to underline the difficulty related to the tasks of estimating and exploiting acoustic echoes to improve indoor audio processing.
    Therefore, this thesis consists only of a first attempting work that lays analytical foundations on how to model such problems, and it can serve as a starting point for new exciting directions.

    \newthought{Keywords:}
    Acoustic echoes, acoustic echo retrieval, room impulse response estimation;
    audio scene analysis, room acoustics;
    audio source separation, room geometry estimation, spatial filtering, sound source localization;
    deep learning, continuos dictionary.
}