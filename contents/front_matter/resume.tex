\chapter*{Résumé en français}\addcontentsline{toc}{chapter}{Résumé en français}


    L'analyse
    \marginpar{
        \footnotesize
        \textbf{Mots-cl\'{e}s:}
        \\Echos acoustiques, récupération des échos acoustiques, estimation des réponses impulsionnelles d'une pièce;
        analyse de scène sonore, acoustique des salles;
        séparation de sources audio, estimation de la géométrie d'une pièce, filtrage spatial, localisation de sources sonores;
        apprentissage profond, dictionnaires continus.
    } de scène audio vise à récupérer des informations utiles à partir d'enregistrements microphoniques.
    La séparation et la localisation de sources sonores sont des exemples de ces problèmes, dans lesquels on s'intéresse à l'estimation du contenu et des positions de multiples sources de son dans un environnement.
    En tant qu'humains, nous effectuons ces tâches sans effort.
    Cependant, pour les ordinateurs et les robots, ces tâches restent des défis à relever.
    L'une des principales limites est que la plupart des technologies disponibles résolvent les problèmes d'analyse de scènes sonores en tenant compte soit de la sémantique du son, soit des informations spatiales.
    \\Le thème central de cette thèse est celui des échos acoustiques: les éléments de la propagation du son faisant le pont entre les informations sémantiques et spatiales des sources sonores.
    En effet, en tant que répétitions d'un signal source, leurs contributions sémantiques peuvent être aggrégées pour améliorer ce signal.
    De plus, comme ils sont issus d'une interaction avec l'environnement, leurs cheminements peuvent être retracés et utilisés pour estimer la géométrie de la scène sonore.
    Sur la base de ces observations, des méthodes récentes en traitement du signal audio tenant compte des échos ont été proposées.
    Deux questions principales se posent : comment estimer les échos acoustiques et comment exploiter leur connaissance?
    \\Ce travail de thèse vise à améliorer l'état de l'art actuel en traitement du signal audio dans des salles selon ces deux axes.
    Il fournit également de nouvelles méthodologies et données pour traiter les échos acoustiques et dépasser les limites des approches actuelles.
    À ces fins, nous présentons tout d'abord deux approches :
    Une nouvelle méthode basée sur le cadre des dictionnaires continus qui ne nécessite pas de réglages de paramètres ni de données d'apprentissage,
    et un modèle d'apprentissage profond estimant le décalage temporel de l'arrivée des premiers échos à l'aide de régularisateurs physiquement motivés.
    En outre, nous présentons un nouvel ensemble de données entièrement annotées, spécialement conçu pour l'estimation d'échos acoustiques et les applications prenant en compte les échos, ouvrant la voie à de futures recherches dans ce nouveau domaine.
    La deuxième partie de cette thèse concerne l'extension de méthodes existantes d'analyse de scènes audio dans leur forme adaptée aux échos.
    \\Le cadre de la NMF multicanale pour la séparation de sources audio, la méthode de localisation SRP-PHAT et la technique de formation de voies (beamforming) MVDR pour l'amélioration de la parole
    étendues à des versions "echo-aware". Ces applications montrent comment un simple modèle d'écho peut conduire à une amélioration des performances.

    \newthought{Cette thèse} souligne la difficulté d'exploiter les échos acoustiques pour améliorer le traitement audio à dans des salles.
    Elle constitue une première tentative de jeter des bases analytiques et méthodologiques unifiées pour résoudre ces problèmes et nous espérons qu'elle serve de point de départ à des de nouvelles recherches prometteuses dans ce domaine.