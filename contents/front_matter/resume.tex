\chapter*{Résumé en français}\addcontentsline{toc}{chapter}{Résumé en français}

{   \small
    L'analyse de la scène audio vise à récupérer des informations utiles à partir d'enregistrements de microphones.
    La séparation et la localisation de la source sonore sont des exemples de ces problèmes, pour lesquels nous nous intéressons à l'estimation du contenu et de la position d'un locuteur.
    En tant qu'humains, nous effectuons ces tâches sans effort. Cependant, pour les ordinateurs et les robots, ces tâches restent des défis à relever.
    L'une des principales limites est que la plupart des technologies disponibles résolvent les problèmes d'analyse des scènes sonores en tenant compte soit de la sémantique du son, soit des informations spatiales.

    \mynewline
    Le thème central de ces thèses est celui des échos acoustiques : les éléments de propagation du son faisant le pont entre les informations sémantiques et spatiales des sources sonores.
    En particulier, comme les répétitions d'une source sonore, leur contribution sémantique peut être intégrée pour améliorer la source sonore.
    Comme ils sont issus de l'interaction avec l'environnement, leur cheminement peut être retracé et utilisé pour estimer la géométrie de la scène sonore.
    Sur la base de ces observations, des méthodes récentes de traitement des signaux audio tenant compte de l'écho ont été proposées.
    Cependant, deux questions principales se posent : comment estimer les échos acoustiques et comment utiliser leurs connaissances ?

    \mynewline
    Ce travail de thèse vise à améliorer l'état actuel de la technique pour le traitement des signaux audio à l'intérieur des bâtiments selon ces deux axes.
    Il fournit également de nouvelles méthodologies et données pour traiter les échos acoustiques et dépasser les limites des approches actuelles.
    À cette fin, nous présentons deux approches :
    Une nouvelle approche basée sur le cadre du dictionnaire continu qui ne repose pas sur des techniques de réglage de paramètres ou de pic de crête ;
    Un modèle d'apprentissage approfondi estimant le décalage temporel de l'arrivée des premiers échos importants à l'aide de régularisateurs physiquement motivés.
    En outre, nous présentons un ensemble de données entièrement annotées, spécialement conçu pour la récupération d'échos acoustiques, une application prenant en compte l'écho pour valider les futures recherches sur l'écho.

    \mynewline
    La deuxième partie de cette thèse concerne l'extension des méthodes existantes d'analyse de scènes audio dans leur forme adaptée à l'écho.
    Le cadre NMF multicanal pour la séparation des sources audio, les méthodes de localisation SRP-PHAT et le formateur de faisceau MVDR pour l'amélioration de la parole
    sont proposés dans leur version "echo-aware". Ces applications montrent comment un simple modèle d'écho peut conduire à une augmentation des performances.

    \noindent Enfin, nous voulons souligner la difficulté liée aux tâches d'estimation et d'exploitation des échos acoustiques pour améliorer le traitement audio en intérieur.
    Cette thèse ne constitue donc qu'une première tentative de travail qui pose des bases analytiques sur la façon de modéliser de tels problèmes, et elle peut servir de point de départ à de nouvelles orientations passionnantes.

    \newthought{Mots-cl\'es:}
    Echos acoustiques, récupération des échos acoustiques, estimation de la réponse impulsionnelle de la pièce ;
    analyse de la scène sonore, acoustique de la pièce ;
    séparation des sources audio, estimation de la géométrie de la pièce, filtrage spatial, localisation de la source sonore ;
    apprentissage approfondi, dictionnaire des continuos.
}