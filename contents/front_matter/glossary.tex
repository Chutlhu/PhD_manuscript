% Diego Di Carlo
% diego.di-carlo@inria.fr

% === GLOSSARY ===
\begin{acronym}[UMLX]
    \acro{AB}{Almost Bent}
    \acro{ACT}{Autocorrelation Table}
    \acro{AES}{Advanced Encryption Standart}
    \acro{ANF}{Algebraic Normal Form}
    \acro{APN}{Almost Perfect Nonlinear}
    \acro{ASIC}{Application Specific Integrated Circuit}
    \acro{BCT}{Boomerang Connectivity Table}
    \acro{BCT}{Boomerang Connectivity Table}
    \acro{CBC}{Cipher Block Chaining}
    \acro{CCA}{Chosen Ciphertext Attack}
    \acro{CFB}{Cipher Feedback}
    \acro{CPA}{Chosen Plaintext Attack}
    \acro{CTR}{Counter}
    \acro{DDT}{Difference Distribution Table}
    \acro{DES}{Data Encryption Standart}
    \acro{DLCT}{Differential-Linear Connectivity Table}
    \acro{DP}{Differential Probability}
    \acro{ECB}{Electronic Codebook}
    \acro{EDP}{Expected Differential Probability}
    \acro{ELP}{Expected Linear Potential}
    \acro{FPGA}{Field Programmable Gate Array}
    \acro{IoT}{Internet-of-Things}
    \acro{KPA}{Known Plaintext Attack}
    \acro{LAT}{Linear Approximation Table}
    \acro{LFSR}{Linear Feedback Shift Register}
    \acro{LP}{Linear Potential}
    \acro{LUT}{Look Up Table}
    \acro{LWC}{Lightweight Crypto}
    \acro{MDS}{Maximum Distance Separable}
    \acro{MEDP}{Maximum Expected Differential Probability}
    \acro{MELP}{Maximum Expected Linear Potential}
    \acro{MILP}{Mixed Integer Linear Program}
    \acro{OFB}{Output Feedback}
    \acro{OTP}{One Time Pad}
    \acro{PRP}{Pseudo\-random Permutation}
    \acro{SLP}{Straight-Line Program}
    \acro{SPN}{Substitution Permutation Network}
    \acro{WSN}{Whitened Swap-Or-Not}

    % --- ch:intro
    \acro{CASA}{Computational Auditory Scene Analysis}

    % --- ch:acoustics
    \acro{SOTA}{State of the Art}
    \acro{GA}{Geometrical (room) acoustics}
    \acro{FEM}{Finite Element Method}
    \acro{BEM}{Boundary Element Method}
    \acro{FDTD}{Finite-Difference-Time-Domain}
    \acro{DWM}{Digital Waveguide Mesh}
    \acro{ISM}{Image Source Method}
    \acro{RIR}{Room Impulse Response}
    \acro{ReIR}{Relative Impulse Response}
    \acro{FIR}{Finite Impulse Resposne}
    \acro{ATF}{Acoustic Transfer Function}
    \acro{AIR}{Acoustic Impulse Response}
    \acro{TF}{Time-Frequency}
    % --- ch:processing
    \acro{SE}{Speech Enhancement}
    \acro{SSS}{Sound Source Separation}
    \acro{SSL}{Sound Source Localization}
    \acro{RooGE}{Room Geometry Estimation}
    \acro{AER}{Acoustic Echo Retrieval}
    \acro{FT}{Fourier Transform}
    \acro{DFT}{Discrete Fourier Transform}
    \acro{DTFT}{Discrete-Time Fourier Transform}
    \acro{STFT}{Short Time Fourier Transform}
    \acro{FFT}{Fast Fourier Transform}
    \acro{RTF}{Relative Transfer Function}
    \acro{ILD}{Interchannel Level Difference}
    \acro{IPD}{Interchannel Phase Difference}
    \acro{ITD}{Interchannel Time Difference}
    \acro{TDOA}{Time Difference of Arrival}
    \acro{AWGN}{Additive White Gaussian Noise}

\end{acronym}

% \makeglossaries
% \addcontentsline{toc}{chapter}{Glossary}
% \printglossary[type=\acronymtype]
% \printglossary
% % \makeglossary

% % \newacronym{MRP}{MRP}{Major Research Paper}
% % \newacronym{YSGS}{YSGS}{Yeates School of Graduate Studies}

% % \newglossaryentry{cookiecutter-latex-ryerson}
% % {
% % 	name={cookiecutter-latex-ryerso}n,
% % 	description={is a personal LaTeX template by Richard Wen for Ryerson University's paper submission requirements}
% % }

% % \newglossaryentry{Ryerson University}
% % {
% % 	name={Ryerson University},
% % 	description={is a Canadian university in Toronto, Ontario}
% % }
