% verso: colophon / impressum
\thispagestyle{empty}
\hphantom{.}
\vfill

\section*{Imprint}

\textit{Echo-aware signal processing for audio scene analysis}\\
Copyright \textcopyright{} 2020 by \theauthor{}.\\
All rights reserved. Printed in France.\\
% Published by the Ruhr-Universität Bochum, Bochum, Germany.

\section*{Colophon}

This thesis was typeset using \LaTeX{} and the \texttt{memoir} documentclass.
It is based on Aaron Turon's thesis \emph{Understanding and expressing scalable concurrency}\footnote{\url{https://people.mpi-sws.org/~turon/turon-thesis.pdf}}, itself a mixture of \texttt{classicthesis}\footnote{\url{https://bitbucket.org/amiede/classicthesis/}} by Andr\'e Miede and \texttt{tufte-latex}\footnote{\url{https://github.com/Tufte-LaTeX/tufte-latex}},
 based on Edward Tufte's \emph{Beautiful Evidence}.\\[0.5\baselineskip]
%
The bibliography was processed by Biblatex.
Graphics and plots are made with PGF/Ti\emph{k}Z, Matplotlib, Mathcha and Excalidraw.\\[0.5\baselineskip]
%
The body text is set 10/14pt (long primer) on a 26pc measure.
The margin text is set 8/9pt (brevier) on a 12pc measure.
\library{Linux Libertine} acts as both the text and display typeface.
Monospaced text uses Jim Lyles's \texttt{Bitstream Vera Mono} (\enquote{Bera Mono}).

\clearpage{}

% % recto: dedication or epigraph
% \thispagestyle{empty}
% \vphantom{.}
% \vfill
% \begin{dedication}
%     A nonna Leti,\\
%     ``Lavati le man e va a lavorare!'';
%     \vfill
%     A nonna Carla,\\
%     ``Mammmmma miaaa, che beo ció!'';
%     \vfill
%     A Giorgia,\\
%     ``Non perdere tempo con 'ste cose!''
% \end{dedication}
% {%
% \flushright{}
% \emph{OAndinasdisaodsnd{sdassaodisn
% dsnaodsudaosnud ssn[dsnsi+oir]\\
% Aaaaaaaaaaaaahhhh}.}\\
% \hfill--- Diego \textsc{Di Carlo}
% }
% \vfill
% \vfill

% % verso: blank
% \clearpage{}


% \begin{dedication}
%     A nonna Leti,\\
%     ``Lavati le man e va a lavorare!'';
%     \vfill
%     A nonna Carla,\\
%     ``Mammmmma miaaa, che beo ció!'';
%     \vfill
%     A Giorgia,\\
%     ``Non perdere tempo con 'ste cose!''
% \end{dedication}
% \blankpage{}
% \clearpage