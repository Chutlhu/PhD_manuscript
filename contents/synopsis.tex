\newcommand{\synopsisChAcoustics}{
    This chapter will build a first important bridge: from acoustics to audio signal processing.
    It first defines sound and how it propagates in the environment~\cref{ch:acoustics:sec:wave}, teasing out the fundamental concepts of this thesis: the echoes.~\cref{ch:acoustics:sec:reflection} and the \RIRdef/~\cref{ch:acoustics:sec:rir}.
    By assuming some approximations, the \RIR/ will be described in all its parts in relation with methods to compute them.
    Finally, in~\cref{ch:acoustics:sec:perception}, how the human auditory system perceives reverberation will be reported.
}
\newcommand{\synopsisChProcessing}{
    Let us now move from the physics to digital signal processing.
    At first in~\cref{sec:processing:model}, this chapter formalizes fundamental concepts of audio signal processing such as signal, mixtures and noise in the time domain.
    In~\cref{sec:processing:domains}, we will present the signal representation that we will use throughout the entire thesis: the \STFTdef/.
    Finally, after assuming the narrowband approximation, in~\cref{sec:processing:rirmodels} some important models for the \RIRdef/ are described.
}
\newcommand{\synopsisChEstimation}{
This chapter amis to provide the reader with knowledge of the state-of-the-art of \AERdef/.
After presenting the \AER/ problem in~\cref{sec:estimation:problem}, the chapter is divided into three main sections:
\cref{sec:estimation:taxonomy} defines the categories of methods thank to which the literature can be clustered and analyzed in detail later in~\cref{sec:estimation:sota}.
Finally, in~\cref{sec:estimation:datametrics} some datasets and evaluation metrics for \AER/ are presented.
}

\newcommand{\synopsisChBlaster}{
This chapter proposes a novel approach for \textit{off-grid} \AER/ from a stereophonic recording of an unknown sound source such as speech.
In contrast with existing methods, the proposed approach, named \BLASTERdef/.
It is built on the recent framework of \CDdef/ and it does not rely on parameter tuning nor peak picking techniques by working directly in the parameter space of interest.
The accuracy and robustness of the method are assessed on challenging simulated setups with varying noise and reverberation levels and are compared to two state-of-the-art methods.
While comparable or slightly worse recovery rates are observed for the task of recovering 7 echoes or more, better results are obtained for fewer echoes and the off-grid nature of the approach yields generally smaller estimation errors.
}

\newcommand{\synopsisChLantern}{
    This chapter
}

\newcommand{\synopsisChSeparake}{
    In this chapter echoes are used for boosting the performances of state-of-the-art approaches in Audio Source Separation.
    At first, in~\cref{sec:separake:intro}, we describe the existing methods, which typically that either ignore the acoustic propagation, nor attempt to estimate it fully.
    Instead, thanks to the geometric interpretation of early echoes, this chapter show how this gives us enough spatial diversity to get a performance boost over the anechoic case.
    The improvements are show for two standard algorithms based on non-negative matrix factorization---one that uses only magnitudes of the transfer functions, and one that also uses the phases.
    The experimental part shows that the proposed approach based on a few echoes beats its vanilla variant, and that with magnitude information only, echoes enable separation where it was previously impossible.
}
