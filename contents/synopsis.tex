\newcommand{\synopsisChAcoustics}{
    This chapter builds a first important bridge: from acoustics to audio signal processing.
    It first defines sound and how it propagates in the environment~\cref{ch:acoustics:sec:wave}, presenting the fundamental concepts of this thesis: echoes in~\cref{ch:acoustics:sec:reflection} and \acfp{RIR} in~\cref{ch:acoustics:sec:rir}.
    By assuming some approximations, \acp{RIR} will be described by parts in relation to to methods to compute them.
    Finally, in~\cref{ch:acoustics:sec:perception}, how the human auditory system perceives reverberation will be reported.
}
\newcommand{\synopsisChProcessing}{
    Let us now move from physics to digital signal processing.
    At first, in~\cref{sec:processing:model}, this chapter formalizes fundamental concepts of audio signal processing such as signals, mixtures, filters and noise in the time domain.
    In~\cref{sec:processing:domains}, we will present a fundamental signal representation that we will use throughout the entire thesis: the \STFTdef/.
    Finally, in~\cref{sec:processing:rirmodels}, some essential models of the \ac{RIR} are described.
}
\newcommand{\synopsisChEstimation}{
    This chapter aims to provide the reader with knowledge in the state-of-the-art of \acf{AER}.
    After presenting the \ac{AER} problem in~\cref{sec:estimation:problem}, the chapter is divided into three main sections:
    \cref{sec:estimation:taxonomy} defines the categories of methods according to which the literature can be clustered and analyzed in details later in~\cref{sec:estimation:sota}.
    Finally, in~\cref{sec:estimation:datametrics} some datasets and evaluation metrics for \ac{AER} are presented.
}

\newcommand{\synopsisChBlaster}{
    This chapter proposes a novel approach for \textit{off-grid} \ac{AER} from a stereophonic recording of an unknown sound source such as speech.
    In order to address some limitation of existing methods, we propose a new approach, named \BLASTER/.
    It builds on the recent framework of \acf{CD}, and it does not rely on parameter tuning nor peak picking techniques by working directly in the parameter space of interest.
    The method's accuracy and robustness are assessed on challenging simulated setups with varying noise and reverberation levels and are compared to two state-of-the-art methods.
    While comparable or slightly worse recovery rates are observed for recovering seven echoes or more, better results are obtained for fewer echoes, and the off-grid nature of the approach yields generally smaller estimation errors.
}

\newcommand{\synopsisChLantern}{
    In this chapter, we will use virtually supervised deep learning models to learn the mapping from microphone recordings to the echoes' timings.
    After presenting a quick overview of deep learning techniques (\ref{sec:lantern:dnn}), we will present a first simple model to estimate the echo coming from a close surface (\ref{sec:lantern:R1}).
    This case is motivated by an application in \acf{SSL}, which will be discussed in detail in~\cref{ch:mirage}.
    Finally, we will present a possible way to achieve a more robust estimation (\ref{ch:lantern:robust}) and discuss a possible way to scale this approach to multiple echoes (\ref{ch:lantern:conclusion}).
}

\newcommand{\synopsisChDechorate}{
    This chapter presents \dEchorate{}: a new database of measured multichannel room impulse response (RIRs) including annotations of early echoes and 3D positions of microphones, real and image sources under different wall configurations in a cuboid room.
    These data provide a tool for benchmarking recent methods in \textit{echo-aware} speech enhancement, room geometry estimation, RIR estimation, acoustic echo retrieval, microphone calibration, echo labeling, and reflectors estimation.
    The database is accompanied by software utilities to easily access, manipulate, and visualize the data and baseline methods for echo-related tasks.
}

\newcommand{\synopsisChApplication}{
    In this chapter, we will present a selection of algorithms and methodologies which we identified as potential beneficiaries of echo-aware additions.
    At first, in section~\cref{sec:application:scenario}, we present a typical scenario that highlights some cardinal problems.
    Then, in section~\cref{sec:application:sota}, state-of-the-art approaches to address these problems are listed and commented, highlighting the relationship with some acoustic propagation models.
    The content presented here serves as a basis for a deeper investigation conducted in each of the following chapters.
}

\newcommand{\synopsisChSeparake}{
    In this chapter, echoes are used for emproving performance of classical Audio Source Separation methods.
    First, we describe existing methods that either ignore the acoustic propagation or attempt to estimate it fully.
    Instead, this work investigates whether sound separation can benefit from the knowledge of early acoustic echoes.
    These echoes are derived from the known locations of a few \textit{image microphones}.
    The improvements are shown for two variants of a method based on non-negative matrix factorization: one that uses only magnitudes of the transfer functions and one that uses the phases as well.
    The experimental part shows that the proposed approach beats its vanilla variant by using only a few echoes and that with magnitude information only, echoes enable separation where it was previously impossible.
}

\newcommand{\synopsisChMirage}{
    This chapter addresses the problem of audio source localization in the context of strong acoustic echoes.
    Classic \ac{SSL} methods are deceived by strong acoustic echoes:
    rather than estimating the location of a true source, they might be fooled by its strong correlated echos.
    Instead, we show early-echo characteristics can, in fact, benefit \ac{SSL}.
    To this end, we introduce the concept of microphone array augmentation with echoes \MIRAGE/, using the model of image microphones presented in the previous chapter.
    In particular, we show that in a simple scenario involving two microphones close to a reflective surface and one source, the proposed approach can estimate both azimuthal and elevation angles, an impossible task assuming an ideal propagation, as classical approaches do.
    Later, the proposed approach is extended to multichannel recording and tested on real data scenario.
}


\newcommand{\synopsisChDecharateApp}{
    This chapter presents two echo-aware applications that can benefit from the dataset \dEchorate.
    In particular, we exemplify the utilization of these data considering two possible use-cases: echo-aware speech enhancement (\cref{sec:dechorateapp:se}) and room geometry estimation (\cref{sec:dechorateapp:rooge}).
    This investigation is conducted using state-of-the-art algorithms described and contextualized in the corresponding sections.
    In the final section (\cref{sec:dechorateapp:conclusion}), the main results are summarized, and future perspectives are presented.
}