\newcommand{\synopsisChAcoustics}{
    This chapter will build a first important bridge: from acoustics to audio signal processing.
    It first defines sound and how it propagates in the environment~\cref{ch:acoustics:sec:wave}, teasing out the fundamental concepts of this thesis: the echoes.~\cref{ch:acoustics:sec:reflection} and the \RIRdef/~\cref{ch:acoustics:sec:rir}.
    By assuming some approximations, the \RIR/ will be described in all its parts in relation with methods to compute them.
    Finally, in~\cref{ch:acoustics:sec:perception}, how the human auditory system perceives reverberation will be reported.
}
\newcommand{\synopsisChProcessing}{
    Let us now move from the physics to digital signal processing.
    At first in~\cref{sec:processing:model}, this chapter formalizes fundamental concepts of audio signal processing such as signal, mixtures and noise in the time domain.
    In~\cref{sec:processing:domains}, we will present the signal representation that we will use throughout the entire thesis: the \STFTdef/.
    Finally, after assuming the narrowband approximation, in~\cref{sec:processing:rirmodels} some important models for the \RIRdef/ are described.
}
\newcommand{\synopsisChEstimation}{
This chapter amis to provide the reader with knowledge of the state-of-the-art of \AERdef/.
After presenting the \AER/ problem in~\cref{sec:estimation:problem}, it is divided into three main sections:
\cref{sec:estimation:taxonomy} defines the categories of methods thank to which the literature can be clustered and analyzed in detail later in~\cref{sec:estimation:sota}.
Finally, in~\cref{sec:estimation:datametrics} some datasets and evaluation metrics for \AER/ are presented.
}