\newthoughtpar{The Derivation of the Acoustic Wave Equation}
Let's starts by considering a infinitesimal volume unit $\volume$ of a fluid or gas (such as air),
whose center of gravity is located at $\positionMicrophone \in \bbR^3$.
Let $\mass$ be the mass of such volume.
By the well-known Newton's second law, applying a force $\forceVec$ to the fluid, its acceleration increase proportionally to $\mass$,
namely:
\begin{equation}
    \label{eq:acoustics:newton}
    \forceVec = \mass \kpderiv[]{\velocity\depSpaceTime}{t}
\end{equation}
\marginpar{%
\footnotesize
Newton's Law II: The alteration of motion is ever proportional to the motive force impress'd;
and is made in the direction of the right line in which that force is impress'd.
\\Orignal: \emph{Lex II: Mutationem motus proportionalem esse vi motrici impressae,
et fieri secundum lineam rectam qua vis illa imprimitur.}
}
where $\velocity(\positionSource, t)$ denotes the volume velocity and $t$ the time $[\si{\second}]$.
The force can be expressed in terms of difference of acoustic pressure $p$ at $\positionMicrophone$ on a surface of the volume, $\surface$, namely
\begin{equation}
    \label{eq:acoustics:pressure}
    \forceVec = - \surface \kbracket{\kgrad{\pressureSpaceTime}}
\end{equation}
where $\kgrad{}$ is is the gradient operator.
\\By combining \cref{eq:acoustics:newton,eq:acoustics:pressure}, we obtain the famous \textit{Euler's equation of motion}:
\begin{equation}
    \label{eq:acoustics:euler}
    \kgrad{\pressureSpaceTime} = - \densityEq \kpderiv[]{\velocity\depSpaceTime}{t}
\end{equation}
where $\densityEq = \frac{\mass}{\surface}$ is the static density of the medium
\footnote{%
\label{fn:acoustics:airconstanc}
Selected physical quantities for air:
\\Air Denisity $\density_{\text{air}} = 1.18 \tfrac{\si{\kilogram}}{\si{\metre^3}}$.
\\Air Gas constant $R_{\text{air}} = 286.9 \tfrac{\si{\joule}}{\si{\kilogram} \si{\mole}}$.
\\Air Adiabatic index $\gamma_{\text{air}} = 1.4$.
\\Speed of sound in air $\speedOfSound_{\text{air}} = 343.1 \tfrac{\si{\metre}}{\si{\second}}$.
}.

\newthought{By the Conservation of Mass principle}, states that, in a deformable medium, the total mass must remain constant.
This principle translates into the \textit{continuity equation}, which its differential form writes
\marginpar{%
    \footnotesize
    $\divergence{} = \kpderiv[]{}{x} + \kpderiv[]{}{y} + \kpderiv[]{}{z}$ is the divergence operator.
}
\begin{equation}
    \label{eq:acoustics:continuity}
    \kpderiv[]{\volumeUnit\depSpaceTime}{t} = V \kbracket{\divergence{\flux\depSpaceTime}}
\end{equation}
where
\begin{itemize}
    \item $\volumeUnit\depSpaceTime$ is the volume variation due to the pressure changing, and
    \item $\flux\depSpaceTime$ is the \textit{flux} of mass $m$ per unit volume (\aka/ flux)
    \sidenote{%
        \footnotesize
        flux is by definition equal to the density times the velocity.
        In math, $\flux\depSpaceTime = \densityEq \velocity\depSpaceTime$.
    }.
\end{itemize}

\newthought{The Polytropic Process Relation} assumed properties on the propagation medium.
Since the exchange of heat in negligible in the acoustic frequencies range,
the whole process can be considered thermodynamically \textit{adiabatic}.
In such scenario, the relation between the total pressure $\calP$ and the total volume $\volume$ is given by
\marginpar{
    \footnotesize
    The dependency upon space and time $\depSpaceTime$ is here omitted for sake of compactness and readability.
}
\begin{equation}
    \label{eq:acoustics:state}
    \calP \volume^\gamma = \const
\end{equation}
where $\gamma$ is the adiabatic index of the medium\sidenote{Cfr.~\cref{fn:acoustics:airconstanc}}.
The total pressure and the total volume consist in a sum of a constant and a variable term,
that is $\calP = P_0 + p$,
$\volume = V_0 + \nu$ respectively.
Considering that $p \ll P_0$ and $\nu \ll V_0$, the time-differential of~\cref{eq:acoustics:state} with respect to time reads
\begin{equation}
    \label{eq:acoustics:statediff}
    \kpderiv[]{\pressureSpaceTime}{t} = - \gamma \frac{P_0}{V_0} \kpderiv[]{\nu\depSpaceTime}{t}
\end{equation}

\newthought{Finally, the Acoustic Wave Equation} can be now derived by combining together
the equation of motion~\eqref{eq:acoustics:euler},
the continuity equation~\eqref{eq:acoustics:continuity},
and the thermodynamic balance of the medium~\eqref{eq:acoustics:statediff}.
In particular the combination of~\cref{eq:acoustics:continuity,eq:acoustics:statediff},
\begin{equation}
    \kpderiv[]{\pressureSpaceTime}{t} = - \gamma P_0 \kbracket{\divergence{\flux\depSpaceTime}}
    ,
\end{equation}
can be differentiated with respect to time $t$ yielding to
\begin{equation}
    \kpderiv[2]{\pressureSpaceTime}{t} = - \gamma P_0 \kbracket{\divergence{\kpderiv[]{\flux\depSpaceTime}{t}}}
    .
\end{equation}
Taking the divergence of each side of the~\cref{eq:acoustics:euler} and remembering the definition of flux, we get
\marginpar{%
    \footnotesize
    The \textit{Laplacian} of a function is equivalent to the divergence of the gradient of that function.
    \\In math, $\knabla^2 x = \divergence{\kgrad{x}}$}
\begin{equation}
    \knabla^2 \pressureSpaceTime = -\densityEq \kbracket{\divergence{\flux\depSpaceTime}}
\end{equation}
The above two equation can be combined leading to
\begin{equation}
    \knabla^2 \pressureSpaceTime = \frac{1}{\speedOfSound^2} \kpderiv[2]{\pressureSpaceTime}{t}
    ,
\end{equation}
which is equivalent to~\cref{eq:acoustics:wave}.
The constant $\speedOfSound$ is the wave speed, in our case the speed of sound.
Notice that it is related to the medium properties through
\begin{equation}
    \speedOfSound^2 = \frac{\gamma P_0}{\densityEq}
    .
\end{equation}

---

The conventional way to solve~\cref{eq:acoustics:helmholtz} is to find a set of Functions $\Psi_l(f, \positionMicrophone)$ for $l = 0, \dots, \infty$
which satisfy the homogenous equation~\cref{eq:acoustics:helmholtz}
for a certain interval and its boundary conditions $\boundariesConditions$.
\\This type of functions are called \textit{characteristic function} or \textit{eigenfunction}
\sidenote{Notice that that~\cref{eq:acoustics:helmholtz} can be written as eigenfunction/eigenvalue equation, \ie/ $\knabla^2 H = -k^2 H$ }
and depends on $\boundariesConditions$.
In general, such kind of function are too difficult to be computed in closed form and needs to be approximated with numerical methods.
Their analytical expression is known only for a few room shapes combined with simple boundary conditions.

Subsequently, the general expression of the Green's function $H(f, \positionMicrophone \mid \positionSource)$ can be expressed as a sum of the eigenfunction weighted on
a coefficient $C_l(f, \positionSource)$ dependent on the source position \citeonly{habets2006room}:

\begin{equation}
    \label{eq:acoustics:eigenfunction}
    H(f, \positionMicrophone \mid \positionSource) =
        \sum_{l=0}^\infty
            C_l(f, \positionSource)
            \Psi_l(f, \positionMicrophone)
\end{equation}

---


Let us assume the simplest possible 3D enclosure: a \textit{shoebox}, \ie/ a cuboid room with perfectly smooth and rigid facets.
Lets define the domain $\calD$ of the problem: the cuboid length $L$, width $W$ and height $H$, that is
\begin{equation}
    \calD = \kset{\positionMicrophone = (x,y,z)}{
        0 \le x \le L_x,\,
        0 \le y \le L_y,\,
        0 \le z \le L_z}
\end{equation}
Given the boundaries $\calB$ of $\calD$, the frequency-domain Green's function associated to \cref{eq:acoustics:helmholtz} is given by
\marginpar{%
    \footnotesize
    $k = \sfrac{2 \pi f}{c}$,
    \\$\ntuple = \ktuple{n_x, n_y, n_z}$,
    \\$n_i \in \bbN \hspace{1em} \kforall[i \in \kbrace{x, y, z}]$,
    \\$V = L_x L_y L_z$ is the room volume,
    \\$\calM = \kset{\ktuple{n_x, n_y, n_z}}{n_i \in N}$.
}
\begin{equation}
    \label{eq:acoustics:greenBounded}
    H(f, \positionMicrophone \mid \positionSource) =
    - \frac{1}{V}
    \sum_{\ntuple \in \calM}
    \frac{\Psi_{\ntuple}(\positionMicrophone)\Psi_{\ntuple}(\positionSource)}{\kappa^2_{\ntuple} - k^2}
\end{equation}%
where
\marginpar{%
    \footnotesize
    $k_i = \sfrac{n_i \pi}{L_i} \hspace{1em} \kforall[i \in \kbrace{x, y, z}]$
}%
\begin{align}
    \kappa_{\ntuple}    &= \ktuple{\frac{n_x \pi}{L}, \frac{n_y \pi}{W}, \frac{n_z \pi}{H}} = \ktuple{k_x, k_y, k_z} \nonumber\\
    \kappa_{\ntuple}^2  &= \abs{\kappa_{\ntuple}}^2 = k_x + k_y + k_z \nonumber\\
    \Psi_{\ntuple}(\pos) &= \cos\kparen{k_x x}\cos\kparen{k_y y}\cos\kparen{k_z z} \label{eq:acoustics:mode}
\end{align}
is the eigenfunction\sidenote{Cfr~\cref{eq:acoustics:eigenfunction}} for the specific shoebox boundaries $\boundariesConditions$~\citeonly{kuttruff2016room}.

Using the exponential expansion of the cosine for~\cref{eq:acoustics:mode}
and using it in~\cref{eq:acoustics:greenBounded}, we obtain

\begin{equation}
    H(f, \positionMicrophone \mid \positionSource) =
    - \frac{1}{8 V}
    \sum_{\ntuple \in \calM}
    \sum_{\ntuple \in \calM}
    \frac{\Psi_{\ntuple}(\positionMicrophone)\Psi_{\ntuple}(\positionSource)}{\kappa^2_{\ntuple} - k^2}
\end{equation}%

---
\newthoughtpar{Relation with the Helmholtz Equation}

Finally, the equation becomes
\begin{equation}
    \label{eq:acoustics:ims:frourier}
    H(f, \positionMicrophone \mid \positionSource) =
        \sum_{p=1}^{8}
            \sum_{\pos=-\infty}^{\infty}
                \frac{1}{4 \pi \norm{\coordinatePermutation_p +  \coordinatePermutation_p}}
\end{equation}

by taking the inverse Fourier Transform, the echo structure becomes explicit.

We can write the final Room Impulse Response $\rir_{ij}(t)$ as follows:
\begin{equation}
    \contMicrophoneSignal(t) = (\rir_{ij} \conv \contSource)(t)
\end{equation}

\begin{equation}
    \rir_{ij}(t) = \sum_{r=0}^{R} \frac{\alpha_r}{4 \pi \tau_r / \cair} \delta \kparen{t - \tau_r}
\end{equation}
where
\begin{itemize}
    \item $\alpha_r \in \kintervcc{0}{1}$ is the attenuation coefficient of the $r$-th reflection
    \item $\tau_r = \norm{\positionMicrophone_\idxMic - \positionSource_\idxEch}$ is the distance between the microphone and the $\idxEch$-th image of source $\idxSrc$.
\end{itemize}