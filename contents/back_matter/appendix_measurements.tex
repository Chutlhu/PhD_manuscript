\chapter{RIR and RT60 measurements}\label{ap:rir}

\newthought{Synopsis} \marginpar{%
\footnotesize
\textbf{Resources:}
\begin{itemize}
    \item \href{https://github.com/Chutlhu/Risotto}{\library{Risotto} \ExternalLink}
    \item \href{https://github.com/maj4e/pyrirtool}{\library{pyrirtool} \ExternalLink}
    \item \href{https://dsp.stackexchange.com/questions/17121/calculation-of-reverberation-time-rt60-from-the-impulse-response}{StackExchange/DSP discussion}
\end{itemize}
} This appendix briefly the procedures used for measuring \acp{RIR} and $\mathtt{RT}_{60}$ in the work described in~\cref{ch:dechorate}.

\section{Measuring RIR}\label{sec:rir:rir}

There are several ways to measure room impulse responses. Some interesting and recent references are:
\begin{itemize}
    \item Farina: author of a paper and a AES tutorial on Measuring with Exponential Sine Chirps~\citeonly{farina2007advancements}.
    \item Igor Szoke: author of the paper \textit{Building and Evaluation of a Real Room Impulse Response Dataset}
    from Czech Republic.
    It explain different pro and cons of different RIR estimation method as well as a protocol for acquiring data.
    \item Kurtruff: well-known author of the book \textit{Room Acoustics}. It explain the method with the Maximum Length Sequence (MLS).
\end{itemize}

Some implementation code for it that I found:

\begin{itemize}
    \item Sharon's Lab MATLAB utilities

\end{itemize}

There are 3 problems, which maybe be related:

\begin{enumerate}
    \item \textit{what is the best probe signal to be generated, hence, emitted?}
    \item \textit{what is the best method to perform deconvolution?}
    \item \textit{how to eventually post-process the estimation?}
    \\Here people use the following:
    band-pass filtering (Blackman window between $80-8000$ Hz) and then
    average (and which kind of average) between multiple repetition.
    Or compute the estimation for the entire signal consisting on
    multiple chirps.
\end{enumerate}


\section{RT60 estimation}\label{sec:rir:rt60}

\newthought{Empirical, or the Sabine's way} just use Sabine's (or Eyring's) equation

\newthought{From the measurement} and from RIR


Framework:
\begin{enumerate}
    \item \textit{obtain the Impulse Response} which can be done in various ways:
    \begin{itemize}
        \item \textbf{Popping a balloon} basically recording
        any impulsive-like signal with broad frequency content
        and omnidirectional characteristic.
        This is the simplest method of obtaining impulse response.
        \item \textbf{Sweep Sine} measurement with a usage of either linear
        or exponential sweep. It allows you to extract many different
        parameters of your system at the same time.
        \item MLS (Maximum Length Sequence) measurement
        with this kind of noise and using Hadamard transform
        to obtain the impulse response.
        Some comparison with sweep sine technique can be found
        \href{https://ev.fe.uni-lj.si/3-2011/Policardi.pdf}{here}.
    \end{itemize}

    \item \textit{filter impulse response in an appropriate frequency band}.
    This is due to fact that reverberation time is a function of position
    in the room and frequency. Obviously, we assume that sound field is
    diffused and at each position in a room decay rate is the same.
    Thus we must still consider dependency on the frequency band.
    For most of the time, Butterworth 3rd order filter is enough
    (although for frequencies below 125 Hz you might experience some problems
    with characteristic not meeting specifications).

    \item \textit{Smooth it}
    \begin{enumerate}
        \item \textit{Make the response as smooth as possible before conversion
        to the logarithmic scale}.
        For that purpose, \textit{Hilbert Transform} is a widely used tool.
        \sidenote{For Python/Matlab: \texttt{abs(hilbert(h))}.}.

        \item \textit{Then it must be smoothed further}. It is optional and one might consider it unnecessary.
        It is based on Moving Average filter of appropriate length $M$ \sidenote{For sampling frequency of $48$ kHz I've used $M=5001$.}.

        \item \textit{The curve $A(t)$ can be further smoothed} for calculations by using the
        \textit{Schroeder Integration method} of your envelope, also known as Inversed Time integration.
        This method gives you maximally flat decay curve and is very easy to implement.
        \sidenote{\texttt{L(td:-1:1)=(cumsum(hA(td:-1:1))/sum(hA(1:td)));}
        \\Mind that limit of integration (td) is equal to $\inf$.It is true when we do not have any environmental noise.
        Otherwise, you will see that decaying sound 'dives into noise level.
        I think that easiest to implement is the one proposed by Lundeby et al.}

    \end{enumerate}

    \item \textit{Compute the Energy Curve} (not a Power) $A(t)$ which is simply the smoothed Hilbert envelope in logarithmic scale
    \begin{equation}
        E(t) = 20 \log_{10}{\frac{A(t)}{\max{A(t)}}}
    \end{equation}

    \item \textit{Compute the $RT_{60}$}. It must be done by performing linear interpolation (read linear regression)
    of of the decay curve (read Schroeder curve) $E(t)$ with linear function: $L=At + B$ on the correct range.
    Then compute the $RT_{60}$ as follows:
    \begin{equation}
        RT60 = \frac{-60}{A}
    \end{equation}
\end{enumerate}

\newthought{Finally, the Limitation of this Method} follows. It is not possible to have always $60$ dB of
dynamic range. It results that this value must be extrapolated from the followings:
\begin{itemize}
    \item $EDT$ (Early Decaying Time): upper limit is $0$ dB and lower is $-10$dB.
    This parameter correlates well with perceived reverberation time.
    \item $T_{10}$:  upper limit must start at $-5$ dB to remove any
    fluctuations and then lower limit is taken to be $-15$dB,
    but it always must be at least $10$dB above the noise floor.
    So in fact you need at least $25$ dB of dynamic range to be able to calculate $T_{10}$
    \item $T_{20}$: upper limit at $-5$dB, lower at $-25$dB.
    Minimum dynamic range needed is $35$dB.
    \item $T_{30}$: upper limit of $-5$dB, lower at $-35$dB, with minimum $45$dB of dynamic range.
\end{itemize}

For $T_{60}$, $75$ dB of dynamic range is required, which in practice is happening very rarely.
\textbf{For very smooth decay these values should be equal}.
\sidenote{Some more detailed analysis of effects can be found in
\href{https://pure.tue.nl/ws/files/3477262/352481346918469.pdf}{Measuring Room Impulse Responses: Impact of the Decay Range on Derived Room Acoustic Parameters}}.
If you will not perform Schroeder integration, then most probably they will diverge a lot.