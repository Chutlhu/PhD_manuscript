\chapter{\library{Mirage}: Echo-aware Sound Source Localization}\label{ch:mirage}

\marginpar{%
    \footnotesize
    \textbf{Keywords:} Sound Source Localization, Image Microphones, Acoustic Echoes, TDOA Estimation.
    \\\textbf{Resources:}
    \begin{itemize}
        \item \href{https://ieeexplore.ieee.org/document/8683534}{Paper}
        \item \href{https://github.com/Chutlhu/mirage}{Code}
        \item \href{https://sigport.org/documents/mirage-2d-sound-source-localization-using-microphone-pair-augmentation-echoes}{Poster}
        \item \href{https://www.youtube.com/watch?v=SfEmwqxxpYg}{HARU Robot presentation}
    \end{itemize}
}

\newthought{Synopsis} \synopsisChMirage

\mynewline
Together with~\cref{ch:lantern}, this chapter describes methods and results published in~\cite{di2019mirage}, which considers only stereophonic recordings.
In this sense, this chapter provides an application for \acs{LANTERN}, the learning-based echo estimation method presented in the above mentioned chapter.
\\Later, the proposed approach was  extended to multi-microphone recordings in a collaboration with Randy Gomez from the Honda Research Institute.
In particular, the method was tested on an autonomous robot platform called \textit{HARU}\citeonly{ackerman2018haru, gomez2018haru}, \marginpar{
    \includegraphics[width=\linewidth]{mirage/haru.jpeg}
    \captionof{figure}{The HARU Robot.}
    \label{fig:mirage:haru}
} consisting of a base and two screens mimicking a face with two eyes (See~\cref{fig:mirage:haru})
The robot is fitted with cameras and a microphone array for visual and audio sensing.
The array is circular with 6.5~cm radius, featuring 7 sensors.
The partner agreed on using its technology to see the impact of echo-aware sound source localization.
% The results of this study were described in an internal technical report~\citeonly{di2019honda}.


\section{Literature review in Echo-aware Sound Source Localization}
Common to most sound source localization approaches reviewed in ~\cref{subsec:application:localization} is the challenge posed by reverberation.
It is typical to observe that \ac{DOAs} estimation degrades with increasing acoustic reflection~\citeonly{chen2006time}.
For these reasons, most sound source localization methods regard reverberation and, in particular, acoustic echoes as a nuisance.
Some prior works, \eg/, \citeonly{rui2004time, chen2006time, zhang2007maximum}, modeled the reverberation as a noise term.
However, the generic model for reverberation does not reduce strong early echoes.
Alternatively, approaches in~\citeonly{weinstein1994iterative, taghizadeh2015spatial, salvati2016sound} attempt to solve \SSL/ by estimating the full \RIRs/, which is known to be very a difficult task.

\mynewline
Instead, echo-aware sound source localization methods take another direction:
they exploit the closed-form relation between echoes timings and audio scene geometry expressed by the \acf{ISM}.
Early works such as~\citeonly{korhonen2008acoustic, ribeiro2010turning, ribeiro2010using, svaizer2011use} uses knowledge form the room geometry to estimated the position of the sound source with respect to the arrays.
This idea was subsequently extended in later works, reducing the amount of prior knowledge required or addressing different applications.
The authors of \citeonly{nakashima2010localization} study the \SSL/ problem in binaural recordings.
To improve localization, they propose to used ad-hoc reflectors as artificial \textit{pinnae} and a simple reflection model.
In the work~\citeonly{krekovic2016echoslam}, the author addresses the problem \ac{SLAM}\sidenote{
    \ac{SLAM} enables the estimation of a moving robot’s position in relation to a number of external acoustic sources.
} using echoes.
The authors of~\citeonly{an2018reflection} leverage on cameras, depth sensors, and laser sensors to identify reflectors and build a corresponding acoustic model that is used later for localizing the sources.
Finally, in a very recent work, the well-known \ac{MUSIC} framework is modified for accounting an echo model in the spherical harmonic representation~\citeonly{birnie2020reflection}

\mynewline
All the above mentioned echo-aware methods are explicitly knowledge-driven, namely, they use closed-form solutions based on physics, acoustics, and signal processing models.
The huge benefit of manipulating ``simple'' models is paid by strong assumption, simplification and hand-crafted feature.
In order to overcome these limitations, data-driven methods have been proposed to address \SSL/ (see~\cref{subsec:application:localization})
This approaches leverage  training datasets to implicitly learn the mapping from audio features to the source position.
Such data can be obtained from real recordings or using physics-based simulators.
These method were showed to overcome some limitations of physics-based model, especially when some assumptions are violated.
However, they are typically trained for specific applications and use-cases (\eg/, arrays geometry, acoustic conditions, \etc/) and fail whenever test conditions strongly mismatch training conditions.

\section{Proposed Approach}
In this work, we propose to combine the best of the two worlds:
\begin{itemize}
    \item to use a data-driven model to estimate sound propagation parameters;
    \item use knowledge-driven model to map such parameters to source's \DOAs/.
\end{itemize}
% To this end, we introduce the framework of \MIRAGEdef/ for \SSL/,

\mynewline
Let us first introduced a simple yet common scenario:
two microphones, one source, and a nearby reflective surface, as illustrated in Fig. \cref{fig:mirage:scene}.
\marginpar{%
    \centering
    \footnotesize
    \includegraphics[trim={50 70 50 150},clip,width=\linewidth]{mirage/scene.pdf}
    \captionof{figure}{%
        Typical setup with one source source recorded by two microphones.
        The illustration shows direct sound path (blue lines) and resulting first-order echoes (orange lines).}
    \label{fig:mirage:scene}
}
This may occur when the sensors are placed on a table or next to a wall.
Striking examples of these scenarios are the smart table-top devices, such as Amazon Echo, Google Home, \etc/.
The reflective surface is assumed to be the most reflective and closest one to the microphones in the environment, generating the strongest and earliest echo in each microphone.
Under this stereophonic \textit{close-surface} model, we ask the following questions.

\questionpar{Can early echoes be estimated from two-microphone recordings of an unknown source?}
To answer this question, we propose to use the class of deep learning models \acs{LANTERN}, presented in~\cref{ch:lantern}.
These models are trained on a close-surface dataset to estimate the first early echoes per channel from audio features.

\questionpar{Can early echoes be used to estimate both the azimuth and elevation angles of the source?}
To answer the second question, we propose the \acf{MIRAGE} framework.
It exploits echoes' time of arrival by expressing them as \acp{TDOA} in a \textit{virtual 4-microphone array} formed by the true microphone pair and its image with respect to the reflective surface.
This model is based on the based on the \textit{image microphones} model (See~\cref{sec:separake:sota}).
Note that answering this question is considered as an \textit{impossible} task in free field stereophonic conditions.

\section{Background in microphone array SSL}\label{sec:background}
In this section, we briefly review some necessary background in microphone array \ac{SSL}.
Let us assume a microphone array of $\numMics$ sensors is placed inside a room and records the sound emitted by one static point sound source ($\numSrcs=1$).
Recalling the signal model presented in~\cref{eq:estimation:signalmodel}, the relationship between the signal $\mic_\idxMic$ recorded by the $\idxMic$-th sensor placed at fixed position $\positionMicrophone_\idxMic$ and the signal $\src$ emitted by the source at fixed position $\positionSource$ writes
\begin{equation}\label{eq:mirage:signalmodel}
    \begin{aligned}
        \tildex_i(t) &= (h_i \convCont \tildes)(t) + \tilden_i(t)\\
        \tildeh_i(t) &= \echoModelTimeSimple{i} + \tilde{\varepsilon}_i(t),
    \end{aligned}
\end{equation}
where $\nse_\idxMic$ denotes possible measurement noise and $\varepsilon_i[k]$ collects later echoes, the reverberation tail, diffusion, and undermodeling noise.
In this work, we will consider only the first strongest echo, therefore $R = 1$.
Note that for $r=0$ denotes the ideal propagation path,
being $\tau^{(0)}_\idxMic$ the ideal propagation path from the source to the $\idxMic$-th microphone, and $\alpha^{(0)}_{\idxMic}$ the air attenuation.
In the remainder of this work, we make the approximation of $\alpha_i^{(\idxEch)}$ being frequency-independent.

\subsection{2-channel 1D-SSL}\label{subsec:mirage:1D-SSL}
\newcommand{\tdoa}{\ensuremath{\tau_\mathtt{TDOA}}}
\newcommand{\aoa}{\ensuremath{\vartheta}}
Let us first consider the case of stereophonic recordings ($\numMics=2$).
Under the far- and free-field assumption, traditional \SSL/ methods use the \acf{TDOA},
\begin{equation*}
    \tdoa \eqdef \tau^{(0)}_2 - \tau^{(0)}_1\quad\text{[second]}
    ,
\end{equation*}
as a proxy for the estimation of the \ac{AOA}, $\aoa$, since:
\begin{equation}\label{eq:mirage:aoa}
    \vartheta = \arccos \kparen{\speedOfSound \: \tdoa \: / \: \distMicMic }\quad\text{[rad]},
\end{equation}
where $\speedOfSound$ is the speed of sound and $\distMicMic$ the inter-microphone distance.

\begin{figure}
    \begin{sidecaption}[]{
        Illustration of the relation between \ac{DOA} and \ac{TDOA} with ones source and two microphone.
        Knowing the distance $\distMicMic$ between the two microphones, simple trigonometry yields the \ac{AOA} $\vartheta$ according to~\cref{eq:mirage:aoa}.
    }[fig:mirage:gcc]
        \includegraphics[width=\linewidth]{mirage/tdoa_microphone.pdf}
    \end{sidecaption}
\end{figure}


\mynewline
Then, \SSL/ reduces to estimating the \ac{TDOA}, which can be done by \ac{CC}-based methods, the \ac{GCC-PHAT} method \citeonly{knapp1976generalized, blandin2012multi}.
Given \STFT/ $\MIC_1$ and $\MIC_2$ of the two microphones signals, the \ac{CC} and \ac{GCC-PHAT} \textit{angular spectra} are defined as:\marginpar{
    \footnotesize\itshape
    The ``generalized'' cross-correlation methods adds weighting functions (\eg/ the phase transfor (PHAT),  or the smoothed coherence transform (SCOT))
    to the \ac{CC}. Their purpose is to improve the estimation of the time delay on specific characteristic of the signal and noise.
    See~\citeonly{chen2006time} for an overview.
}
\begin{equation}\label{eq:mirage:cc}
    \Psi_\mathtt{CC}(\tau) = \sum_{k,l} \MIC_1[k,l] \MIC_2^{*}[k,l] \cste^{-\csti 2  \pi f_k \tau}
    ,
    \end{equation}
\begin{equation}\label{eq:mirage:gccphat}
    \Psi_\mathtt{PHAT}(\tau) = \sum_{k,l}\frac{\MIC_1[k,l] \MIC_2^{*}[k,l]}{\kvbar{ \MIC_1[k,l] \MIC_2^{*}[k,l] }} \cste^{-\csti 2  \pi f_k \tau}
    ,
\end{equation}
where $*$ denotes the complex conjugate, $[k,l]$ indexes a \ac{TF} bin of the \STFT/ and $f_k$ in the $k$-th natural frequency in the \STFT/.
\\The weighting function $1 / \kvbar{ \MIC_1[k,l] \MIC_2^{*}[k,l]}$ is called \acf{PHAT}.
Such function was introduce to suppress the source's autocorrelation component from the angular spectrum.

\mynewline
Then, the \ac{TDOA} estimate is given by
\begin{equation*}
    \hat{\tau}_\mathtt{TDOA} = \arg \underset{\tau}{\max} \; \Psi(\tau)
    ,
\end{equation*}
with $\Psi$ begin either $\Psi_\mathtt{CC}(\tau)$ or $\Psi_\mathtt{PHAT}(\tau)$.
Note that these functions can also be expressed directly as a function of the \ac{AOA} using \eqref{eq:mirage:aoa}, hence the term \textit{angular spectrum}.
Despite the theoretical limitation of \ac{CC}-based methods presented in~\citeonly{chen2006time}, they are known to work well practical typical scenarios.
Moreover, it was showed to be state-of-the-art for \ac{SSL} in a large benchmark study~\citeonly{blandin2012multi}.

\subsection{Multichannel 2D-SSL}\label{subsec:mirage:2D-SSL}
When more microphones are available and the microphones array is compact and not linear\sidenote{
    In case of all the microphones are complanar, the angle can be estimated up to ``up-down'' umbigity.
}, 2D-\ac{SSL} can be envisioned.
A possible approach is to use 1D-\ac{SSL} on all pairs and combine their results, a principle which was successfully applied in the \acf{SRP-PHAT} method \citeonly{dibiase2001robust}.


\mynewline
\begin{figure}
    \begin{sidecaption}[t]{
        Illustration of the different \acp{DOA} at each microphone pairs listening one sound source.
        Knowing the position of the microphone, the angle with respect to a reference point can be deduced in closed-form.
    }[fig:mirage:gcc]
        \includegraphics[width=\linewidth]{mirage/srp-phat_aggregation.pdf}
    \end{sidecaption}
\end{figure}
The \ac{SRP-PHAT} methods returns the source's \ac{DOA}, namely the pair azimuth and elevation $(\theta, \phi)$, by estimating \acp{TDOA} from each microphone pairs.
In order to achieve this, it requires the geometry of the microphone array to be known.
In a nutshell, this algorithm aims to estimate a \textit{global angular spectrum} $\Psi_{\mathtt{SRP}}(\theta,\phi)$ in the polar coordinates system with respect to reference point in the array, typically its barycenter.
This function will exhibit a local maximum in the direction of the active source.
\\The algorithmic can be exemplified in the following steps:\marginpar{
    \footnotesize\itshape
    See \href{http://bass-db.gforge.inria.fr/bss_locate/}{\library{MBSSLocate}\ExternalLink} for a free MATLAB implementation and comprehensive documentation of this algorithm.
}
\begin{enumerate}
    \item a global grid of \acp{DOA} candidates is defined according to a desired resolution and computational load;
    \item for each pair of microphones, a local set of \ac{AOA} (hence, \acp{TDOA}) is defined based on the above chosen \acp{DOA} and the input geometry;
    \item a TDOA-based algorithm (\eg/ \ac{GCC-PHAT}) is used to compute the associated local angular spectrum;
    \item all the local contributions (a collection of local $\Psi_\mathtt{GCC}(\tau)$) are geometrical aggregated and interpolated back to the global \ac{DOA} grid to form $\Psi_{\mathtt{SRP}}(\theta,\phi)$;
    \item the \acs{DOA}(s) maximizing $\Psi_\mathtt{SRP}$ is (are) used as estimate (in case of multiple sources).
\end{enumerate}
% This algorithm can be seen under the \textit{divide-and-conquer paradigm}:
% ``at the leaves'', the \ac{GCC-PHAT} method provides \ac{TDOA} for each microphone pair;
% the ``merge'' operation consists in aggregating \ac{TDOA} defined on a different axis based on the knowledge of the array geometry.
% Finally, we stress that this algorithm is independent of the method used to estimate the \ac{TDOA}.

\section{Microphone Array Augmentation with Echoes}\label{sec:mirage:mirage}
We now introduce the proposed concept of \MIRAGEdef/.
The \RIRs/ in~\cref{eq:mirage:signalmodel} follows to the well known \acf{ISM}, where reflections are treated as mirror images of the true source with respect to reflective surfaces, emitting the same signal.
\marginpar{%
\centering
\footnotesize
    \includegraphics[trim={90 75 40 50},clip,width=\linewidth]{mirage/mirage.pdf}
    \captionof{figure}{%
        Illustration of the images $\mathring{\positionMicrophone}_1$ and $\mathring{\positionMicrophone}_2$ of microphones $\positionMicrophone_1$ and $\positionMicrophone_2$ in the presence of a reflective surface and a source.
        Blue lines correspond to direct paths, orange lines correspond to echo paths.}
    \label{fig:mirage:mirage}
}
We will employ here a less common but equivalent interpretation of \ISM/, namely, the image-microphone (IM) model.
As illustrated in Fig.~\cref{fig:mirage:mirage}, image microphones are mirror images of the true microphones with respect to reflective surfaces.
In this view, the echoic signal received at a true microphone is the sum of the anechoic signals received at this microphone and its images.
If we consider the virtual array consisting of both true and image microphones, multiple microphone pairs are now available.
For each of them, it is then possible to define a corresponding time difference of arrival.
Among them, we will refer to the one between the two real microphones as \acs{TDOA}, the one between the two image microphones as \ac{iTDOA} and the one between the first microphone and its image as \ac{TDOE}.
Therefore, we have:\marginpar{%
    \centering
    \footnotesize
    \includegraphics[trim={82mm 0 0 0},clip,width=\linewidth,height=7cm]{mirage/rirs.pdf}
    \captionof{figure}{%
        Superposition of two \acp{RIR} and visualization of time difference of arrival between direct paths (\ac{TDOA}), first echoes (\ac{iTDOA}) and direct path and first echo (\ac{TDOE}).}
    \label{fig:mirage:rirs_tdoa}
}%
\begin{align}
\tau_\mathtt{TDOA}  &= \tfrac{1}{c} \norm{\positionMicrophone_2 - \positionSource} - \tfrac{1}{c} \norm{\positionMicrophone_1 - \positionSource} = \tau_2^{(0)} - \tau_1^{(0)},\\
\tau_\mathtt{iTDOA} &= \tfrac{1}{c} \norm{\mathring{\positionMicrophone}_2 - \positionSource} - \tfrac{1}{c} \norm{\mathring{\positionMicrophone}_1 - \positionSource} = \tau_2^{(1)} - \tau_1^{(1)},\\
\tau_{\mathtt{TDOE}_1}  &= \tfrac{1}{c} \norm{\mathring{\positionMicrophone}_1 - \positionSource} - \tfrac{1}{c} \norm{\positionMicrophone_1 - \positionSource} = \tau_1^{(1)} - \tau_1^{(0)}
\end{align}
where $\mathring{\positionMicrophone}_i$ denotes the position of the image of $\positionMicrophone_i$ with respect to the reflector.
Note that $\tau_{\mathtt{TDOE}_2} = \tau_\mathtt{iTDOA} + \tau_{\mathtt{TDOE}_1} - \tau_\mathtt{TDOA}$.
These three quantities are directly connected to \RIRs/, as illustrated in~\cref{fig:mirage:rirs_tdoa}.
Let us define $V = \klist{\tau_{\mathtt{TDOA}}, \tau_{\mathtt{iTDOA}}, \tau_{\mathtt{TDOE}_1}}\in\mathbb{R}^3$.
% Note that they are the same estimated by the \acs{LANTERN} \ac{DNN} presented in~.

\mynewline
To learn the parameters $V$, we consider the \acs{LANTERN} data-driven approaches presented in~\cref{ch:lantern}.
It consists in a class of \ac{DNN} architectures trained to perform multi-target regression from the input audio features \acf{ILD} and \acf{IPD} to to the parameters $V\in\bbR^3$.
In fact, those models were proposed exactly to address the task of estimating the first and strongest echo per channel in a close-surface scenario.

\mynewline
Following the 2D-\SSL/ scheme described in \cref{subsec:mirage:2D-SSL} and given the virtual microphone-array geometry (which depends on the relative position of microphones to the surface), $V$ could in principle be used to estimate the 2D directional of arrival of the source.
To this end, the 3 real-value estimates in $V$ need to be converted into local angular spectra (see~\cref{subsec:mirage:2D-SSL}).
In the first investigation of~\citeonly{di2019mirage}, we proposed to use Gaussians functions centred at the estimates $\hat{V}$ with variances equal to the prediction errors made by the \ac{DNN} on the validation set.
Later, as explained in~\cref{sec:lantern:robust}, we modified the networks in order to estimated both these means and the variances.

% In the~\cref{ch:lantern}, we introduced a learning-based method to estimate $V$ using audio features obtained from only two microphones.
% Given a microphone pair, the local maximum of angular spectrum $\Psi_\text{PHAT}$ corresponds to the \ac{TDOA}.
% Moreover, peaks corresponding to the early reflection are also present.
% \cref{fig:mirage:noise_ang_spec} shows the $\Psi_\mathtt{CC}$'s and the $\Psi_\mathtt{PHAT}$'s angular spectra for synthetic data where the source signal is noise or speech for all the pairs of the HARU's circular microphone array.
% The location of the quantities in $V$ are highlighted with vertical dotted lines.
% Theoretically, when only the first reflection are considered ($K=1$), the position of the peaks in the angular spectra correspond to
% $\tau_\mathtt{TDOA}$, $\tau_\mathtt{iTDOA}$, $\tau_\mathtt{TDOA} - \tau_{\mathtt{TDOE}, 1}$, and $\tau_\mathtt{TDOA} + \tau_{\mathtt{TDOE}, 2}$.
% It is important to note that for speech signals, $\Psi_\text{PHAT}$ removes the auto-correlation part in order to promote a sharp peak at the position of the \ac{TDOA}.
% Since acoustic echoes increase the auto-correlations of the signal in one microphones, the \acs{PHAT} transform tends to lower their contribution, so that their peaks are not distinguishable from spurious ones.
% \begin{figure}
%     \begin{fullwidth}
%         \centering
%         \includegraphics[trim={0mm 0 0mm 0},clip,width=\linewidth]{mirage/echo_hunting_broadband.pdf}
%         \includegraphics[trim={0mm 0 0mm 0},clip,width=\linewidth]{mirage/echo_hunting_speech.pdf}
%         \caption{
%             Angular Spectra $\Psi_\mathtt{CC}$ and $\Psi_\texttt{PHAT}$ for different pairs of microphone in the HARU array using synthetic \RIRs/ and \textit{white noise} (top) and \textit{speech} (botton) signal .
%             Vertical lines mark the positions of  $\tau_\mathtt{TDOA}$ (red), $\tau_\mathtt{iTDOA}$ (green), $\tau_\mathtt{TDOA}-\tau_\mathtt{TDOE,1}$ (yellow) and $\tau_\mathtt{TDOA}+\tau_\mathtt{TDOE,2}$ (yellow) are marked with vertical lines.
%             The black vertical lines correspond to the maximum TDOA given the pair distance, \textit{i.e.} corresponding to the AOA $ = \{0, 2\pi\}$}
%         \label{fig:mirage:noise_ang_spec}
%     \end{fullwidth}
% \end{figure}


\section{Experimental Results}\label{sec:mirage:exp}
In this section we will report the experimental results for the proposed approach.
At first we will consider the continuation of the investigation started in~\cref{sec:lantern:simple}.
Later, the approach is extended to real-multichannel recordings using the HARU robot.

\subsection{2-channel scenario}
Here we compare the \ac{MIRAGE} approach using the \acs{LANTERN} \ac{MLP} (MIRAGE-MLP) learning model.
A free and open-source Matlab implementation of \ac{SRP-PHAT}\sidenote{\label{note:mbss}\href{http://bass-db.gforge.inria.fr/bss_locate/}{MBSSLocate\ExternalLink}} is used to aggregate local angular spectra obtained from the \ac{DNN}'s output.
In this stereophonic scenario, the baseline method \ac{GCC-PHAT} was not considered since it can access only the true microphone signals.
Therefore, it is unable to perform 2D-SSL.

\mynewline
The dataset, the implementation, and training procedure are the same as in~\cref{subsec:lantern:simple:mpl:results}.
The methods are both tested on the same synthetic test data featuring white noise (wn), and speech (sp) signals convolved with \acp{RIR} generated by the acoustic simulator~\citeonly{schimmel2009fast}.
Moreover, some recordings are perturbed by external white noise at 10~dB \ac{SNR} (wn+n, sp+n, respectively).
A sphere sampling with $\ang{0.5}$ resolution and coordinates $\theta \in [-179, 180]$ and $\phi \in [0, 90]$ is used for the \c{DOA} search.

\begin{table}[t]
    \begin{sidecaption}[DoA estimation]{%
        Mean angular error in degree (with accuracies ($\%$)) for 2D SSL (azimuth and elevation)
        with $\ang{10}$ and $\ang{20}$ tolerance.}[tab:mirage:doa]
        \small
        \centering
        \begin{tabular*}{\linewidth}{@{\extracolsep{\fill}}cl|cc|cc@{}}
        \toprule
        \ac{DOA}        &            &  \multicolumn{2}{c|}{ACCURACY}    &   \multicolumn{2}{c}{ACCURACY} \\
                        &            &  \multicolumn{2}{c|}{$<\ang{10}$} &   \multicolumn{2}{c}{$<\ang{20}$} \\
                        &    Input   &  $\theta$ &  $\phi$ &  $\theta$ &  $\phi$ \\
        \midrule
        MIRAGE &  wn    &   4.5 (59) &  3.9 (71) &   6.8 (79) &   5.9 (88) \\
        MIRAGE &  wn+n  &   4.4 (18) &  5.5 (26) &   9.4 (35) &  11.1 (66) \\
        MIRAGE &  sp    &   4.6 (45) &  4.8 (59) &   8.1 (71) &   7.2 (83) \\
        MIRAGE &  sp+n  &   5.2 (17) &  5.9 (12) &  10.7 (38) &  12.3 (43) \\
        \bottomrule
        \end{tabular*}
    \end{sidecaption}
\end{table}

\mynewline
Table~\cref{tab:mirage:doa} report the performance of the full MIRAGE-MLP 2D-SSL pipeline.
Within a tolerance of $\ang{20}$, the MIRAGE-MLP model allows estimation of both azimuth and elevation of the target source.
However, since the two microphones were free to move in our dataset, the inclinations of the true and image pairs are rarely flat.
While this helps elevation estimation, it reduces the accuracy of predicting the right azimuth.
While external noise is again decreasing the accuracy dramatically, it is interesting to notice that our \ac{DNN} model trained and validated with white noise sources somewhat generalizes to speech sources.

\mynewline
With this work, we demonstrated how a simple echo model could allow 2D SSL with only two microphones, using simulated data.
In the following section, we will discuss a real-world application of the \ac{MIRAGE} framework to larger microphone arrays.

\newcommand{\MIRAGECNN}{\ensuremath{\mathtt{MIRAGE-CNN}_{\calV_\calN}}}
\subsection{Multi-channel synthetic-data scenario}
In this section, we compare the \ac{SRP-PHAT} algorithm (using GCC-PHAT for TDOA estimation, using the baseline implementation of Sidenote~\ref{note:mbss}) with the proposed approach, \MIRAGE/, on multichannel synthetic data generated with the Python library \library{pyroomacoustics}\sidenote{\href{https://github.com/LCAV/pyroomacoustics}{\library{pyroomacoustics\ExternalLink}}}.
This time we use the more robust \ac{DNN} model based on \ac{CNN} trained with the Gaussian-based log-likelihood of~\cref{eq:lantern:gausslog}, dubbed here as $\MIRAGECNN$.
We recall that the model is specifically trained and validated on white noise sources.
\\The data are generated to match the design of the HARU's microphone array placed on top of a table:
the 7 microphones are at most 30~cm from the close-surface, placed $13$ cm from each other; the other walls' absorption coefficients are uniformly sampled in $(0.5, 1)$, and the one of the close-surface is in $(0, 0.5)$.
200 different audio scenes have been generated including both white noise (wn) and speech source (sp) signals.
Moreover, \ac{AWGN} noise was added to reach 10~dB \ac{SNR}.
The whole dataset design is similar to the one presented in~\cref{subsec:lantern:mlp}.
For \ac{SRP-PHAT}, default parameters where used: a sphere sampling with $\ang{0.5}$ resolution and coordinates $\theta \in [-179, 180]$ and $\phi \in [0, 90]$ for the \c{DOA} search.


\newthoughtpar{2D-SSL estimation on synthetic data}
The performances are considered in terms of mean angular errors $\pm$ standard deviation in degree for both azimuth and elevation.
2D-\ac{DOA} estimation errors using the proposed approach and \ac{SRP-PHAT} are presented in~\cref{tab:mirage:ssl_syth}.
For white noise source signals, \ac{SRP-PHAT} has better performances for both elevation and azimuth, but the proposed approach outperforms the baseline when the emitted signal is speech.
It is interesting to notice that our DNN model trained and validated with white noise sources somewhat generalizes to speech sources.

\begin{table}[t]
    \begin{sidecaption}[]{
        Mean squared errors and standard deviation in degrees for azimuth ($\theta$) and elevation ($\phi$). In bold the best records.
    }[tab:mirage:ssl_syth]
    \centering
    \small
    \begin{tabular*}{\linewidth}{@{\extracolsep{\fill}}lllllll@{}}
        \toprule
        & &  signal &  Error $\theta$  &  Error $\phi$ \\
        \midrule
        & \MIRAGECNN &   wn+n &  1.29 $\pm$   1.17 &   2.30 $\pm$   3.35 \\
        & SRP-PHAT &   wn+n &  \textbf{0.49 $\pm$   0.6}1 &   \textbf{1.70 $\pm$   1.42} \\
        \midrule
        & \MIRAGECNN &  sp+n & \textbf{ 9.51 $\pm$ 15.84} &  \textbf{12.26  $\pm$  12.20} \\
        & SRP-PHAT &  sp+n & 35.27 $\pm$  54.57 &  15.10 $\pm$  16.67 \\
        \bottomrule
    \end{tabular*}
    \end{sidecaption}
\end{table}

\mynewline
In~\cref{fig:mirage:synth_ssl_noise,fig:mirage:synth_ssl_speech}, the \ac{DOA} estimation results are reported as scatter-plots in a prediction-vs-ground-truth plane.
When the test data contains noise, both the methods perform the same.
However, when speech data are considered, \MIRAGECNN{} outperforms the baseline.
Unfortunately, the performance of both of the method drops for elevation estimation.

\begin{figure}[t]
    \begin{sidecaption}[]{
        Scatter-plots (predictions-vs-ground-truth) for DOA estimation (azimuth and elevation) on synthetic data when the the source signal is \textbf{noise}.
        The color map corresponds to different SNR level [dB] in the data.
    }[fig:mirage:synth_ssl_noise]
    \centering
    \subfloat[azel_srp][2D-SSL with SRP-PHAT on noise data]{
        \centering
        \includegraphics[width=\linewidth]{mirage/scatter_azel_estim_SRP_noise.pdf}
        \label{fig:mirage:synth_ssl_noise_srp}
    }
    \hfill
    \subfloat[azel_mirage][2D-SSL with  MIRAGE on noise data]{
        \includegraphics[width=\linewidth]{mirage/scatter_azel_estim_MIRAGE_noise.pdf}
        \label{fig:mirage:synth_ssl_noise_mirage}
    }
    \end{sidecaption}
\end{figure}

\begin{figure}[t]
    \begin{sidecaption}[]{
        Scatterplots (predictions-vs-ground-truth) for DoA estimation (azimuth and elevation) on synthetic data when the the source signal is \textbf{speech}.
        The color map corresponds to different SNR level [dB] in the data.
    }[fig:mirage:synth_ssl_speech]
    \centering
    \subfloat[azel_srp][2D-SSL with SRP-PHAT on speech data]{
        \centering
        \includegraphics[width=\linewidth]{mirage/scatter_azel_estim_SRP_speech.pdf}
        \label{fig:mirage:synth_ssl_speech_srp}
    }
    \hfill
    \subfloat[azel_mirage][2D-SSL with  MIRAGE on speech data]{
        \includegraphics[width=\linewidth]{mirage/scatter_azel_estim_MIRAGE_speech.pdf}
        \label{fig:mirage:synth_ssl_speech}
    }
    \end{sidecaption}
\end{figure}

% \subsection{Multi-channel real scenario}
% \begin{figure}[t]
%     \begin{fullwidth}
%     \centering
%     \subfloat[mu_spkr][HARU Array]{
%         \includegraphics[width=0.32\textwidth]{figures/mirage/1_room_exp}}
%     \hfill
%     \subfloat[mu_spkr][Reflective table]{
%         \includegraphics[width=0.32\textwidth]{figures/mirage/2_room_exp}}
%     \hfill
%     \subfloat[mu_univ][Detail]{
%             \includegraphics[width=0.32\textwidth]{figures/mirage/3_room_exp}}
%     \label{fig:mirage:room_exp}
%     \caption{Picture of the room and setup for recording real multichannel data with the HARU circular microphone array.}
%     \end{fullwidth}
% \end{figure}
% In this section, we compare the \ac{SRP-PHAT} algorithm (using GCC-PHAT for TDOA estimation) with the proposed approach, \MIRAGE/, on real multichannel recordings.
% This time we use the more robust \ac{DNN} model based on \ac{CNN} trained with the Gaussian-based log-likelihood of~\cref{eq:lantern:gausslog}, dubbed here as $\MIRAGECNN$.
% The real multichannel data were recorded with the HARU microphone circular array ($\numMics=7)$.
% The experiments were performed in a big office room 10~m $\times$ 15~m $\times$ 3~m with a reverberation time around 0.2 seconds.
% The HARU was placed on top of a table with a height of 0.10 m to simulate the close-reflector scenario, used to train the \ac{DNN} model.
% The room setup is shown in~\cref{fig:mirage:room_exp}.
% Two loudspeakers were used the emit one anechoic and normalized utterance from the TIMIT dataset and 10 seconds of white noise.
% The dataset consists of 5 different azimuthal positions and for each of them 2 different elevations yielding 10 different locations in space.
% The geometry of the setup was annotated using metric tape measures.


% \newthoughtpar{TDOA estimation and 2D-SSL performances on real data}
% As reported in~\cref{tab:mirage:tdoa_real}, the two methods are comparable both for speech and noise emitted signals even if \ac{SRP-PHAT} performs slightly better.
% In~\cref{fig:mirage:tdoa_real_pairs} the error on \ac{TDOA} estimation is shown of each microphone pair of the HARU.
% It can be seen that the error and the deviation are not homogeneous among the pairs.
% This might be due to some perturbations of the array's microphone positioning: the \ac{SRP-PHAT} method is only successful if the array's geometry is perfectly known \textit{a priori}.
% However, little misplacement leads to local distortion in the input angular spectra\sidenote{
%     In\citeonly{salvati2018exploiting}, the authors address this problem integrating \ac{SRP-PHAT} with a \ac{CNN} together.
% }.
% Moreover, the proposed approach needs the height of the robot (of each pair) as additional information, and again some perturbation can affect performances.

% \begin{table}[h]
%     \begin{sidecaption}[]{
%         Evaluation metrics (\ac{nRMSE}, \ac{RMSE}, \ac{STD}) for TDOA estimation and empirical computational times for different source signals.
%         Boldness denotes the best records.
%     }[tab:mirage:tdoa_real]
%     \centering
%     \small
%     \begin{tabular*}{\linewidth}{@{\extracolsep{\fill}}lllllll@{}}
%         \toprule
%         & &     signal &     nRMSE & RMSE &  STD &      time \\
%         \midrule
%         & \MIRAGECNN  &       noise &  0.26 &  0.94 &  0.57 &  0.26 \\
%         & SRP-PHAT &      noise &  0.20 &  0.62 &  0.57 &  6.48 \\
%         \midrule
%         & \MIRAGECNN &       speech &  0.40 &  1.38 &  0.99 &  0.24 \\
%         & SRP-PHAT &     speech &  0.32 &  0.82 &  1.08 &  4.02 \\
%         \bottomrule
%         \end{tabular*}
%     \end{sidecaption}
% \end{table}

% \begin{figure}
%     \begin{sidecaption}[]{
%         Violin-plots of the TDOA estimation errors (in samples) versus the microphone pairs in HARU for two different source signals on real data
%     }[fig:mirage:tdoa_real_pairs]
%         \centering
%         \includegraphics[trim={10 0 5 5},clip,width=\linewidth]{mirage/violinplot_tdoa_vs_pair_real.pdf}
%     \end{sidecaption}
% \end{figure}

% \mynewline
% Finally, we evaluate the performance of the methods for the 2D-\ac{SSL} task.
% The results in terms of \ac{RMSE} and standard deviation are shown in~\cref{tab:mirage:ssl_real}.
% When the signal emitted is speech, the CNN-based method outperforms the \ac{SRP-PHAT}.
% However, the latter seems to perform better for noise signals, even if comparable error margins are observed.

% \begin{table}[h]
%     \begin{sidecaption}[]{
%         Mean squared errors and standard deviations in degrees for estimation of azimuth ($\theta$) and elevation ($\phi$).
%         In bold the best records.
%     }[tab:mirage:ssl_real]
%     \centering
%     \small
%     \begin{tabular*}{\linewidth}{@{\extracolsep{\fill}}lllll@{}}
%         \toprule
%         &          &  signal &  Error $\theta$  &  Error $\phi$ \\
%         \midrule
%         & \MIRAGECNN   &   noise &   1.29 $\pm$   1.17 &     2.30 $\pm$  3.35 \\
%         & SRP-PHAT &   noise &   \textbf{0.49 $\pm$   0.61} &     \textbf{1.70 $\pm$  1.42} \\
%         \midrule
%         & \MIRAGECNN   &  speech & \textbf{  9.51 $\pm$  15.84} &    \textbf{12.26 $\pm$ 12.20} \\
%         & SRP-PHAT &  speech &  35.27 $\pm$  54.57 &    15.10 $\pm$ 16.67 \\
%         \bottomrule
%     \end{tabular*}

%     \end{sidecaption}
% \end{table}

% \mynewline
% \Cref{fig:mirage:real_ssl_noise,fig:mirage:real_ssl_speech} illustrate the distribution of the prediction of the methods with respect to the ground-truth in the azimuth-vs-elevation planes.
% We see that the predictions do not match the ground-truth properly.
% SRP-PHAT seems to overestimate elevation while predicting well the azimuth, especially for noise signals.
% MIRAGE seems to return more reasonable azimuth-elevation pairs.
% However, the elevation prediction seems to be almost constant across the points.
% Moreover, it seems that there is a constant offset or deviation, especially for azimuth prediction, suggesting that our real-data annotation was perhaps no accurate enough.

% \begin{figure}[h]
%     \begin{sidecaption}[]{
%         Scatterplots (azimuth-vs-elevation) for DOA estimation on real data when the the source signal is \textbf{noise}.
%         The colors indicate reference points (blue) and predicted ones (orange).
%     }[fig:mirage:real_ssl_noise]
%     \centering
%     \subfloat[azel_srp][2D-SSL with SRP-PHAT on real noise data]{
%         \centering
%         \includegraphics[width=0.47\linewidth]{mirage/scatterplot_SRP_broadband.pdf}
%         \label{fig:mirage:real_ssl_noise_srp}
%     }
%     \hfill
%     \subfloat[azel_mirage][2D-SSL with MIRAGE on real noise data]{
%         \centering
%         \includegraphics[width=0.47\linewidth]{mirage/scatterplot_MIRAGE_broadband.pdf}
%         \label{fig:mirage:real_ssl_noise_mirage}
%     }
%     \end{sidecaption}
% \end{figure}

% \begin{figure}[h]
%     \begin{sidecaption}[]{
%         Scatterplots (azimuth-vs-elevation) for DoA estimation on real data when the source signal is \textbf{speech}.
%         The colors indicate reference points (blue) and predicted ones (orange).
%     }[fig:mirage:real_ssl_speech]
%     \centering
%     \subfloat[azel_srp][2D-SSL with SRP-PHAT on real noise data]{
%         \centering
%         \includegraphics[width=0.47\linewidth]{mirage/scatterplot_SRP_speech.pdf}
%         \label{fig:mirage:real_ssl_speech_srp}
%     }
%     \hfill
%     \subfloat[azel_mirage][2D-SSL with MIRAGE on real noise data]{
%         \centering
%         \includegraphics[width=0.47\linewidth]{mirage/scatterplot_MIRAGE_speech.pdf}
%         \label{fig:mirage:real_ssl_speech_mirage}
%     }
%     \end{sidecaption}
% \end{figure}


\section{Conclusion}
This chapter demonstrated how a simple echo model could boost an SSL algorithm in strongly echoic scenarios when microphones are placed close to a reflector.
Instead of integrating the physical equation into their algorithm, we proposed to use a successful algorithm for multichannel SSL on the virtual array created accounting for image microphone.
In order to create such an array, the echoes' parameters need to be estimated.
To this end, we use the learning-based acoustic echo retrieval methods proposed in~\cref{ch:lantern}.
\\Preliminary results on synthetic data for stereophonic recordings prove the effectiveness of the proposed approach.
However, results obtained on real data reveal that the task is still very challenging for both the proposed and baseline methods.
Considering the current knowledge, this is the first time an echo-aware method combines both knowledge-driven and data-driven for sound source localization.
The learning approach could still be significantly improved by considering other acoustic features (such as advance \ReTF/ methods), other architectures, and other challenges.
For instance, handling the missing frequencies of the speech while training on a broadband signal such as in~\citeonly{gaultier2017vast},
using physical-driven regulation such as in~\citeonly{nabian2020physics} and make the learning step independent to the array geometry.
Moreover, further investigation is needed to strengthen these results and further improve the robustness of the learned mapping, for which the \dEchorate may be used as a valuable testing dataset.