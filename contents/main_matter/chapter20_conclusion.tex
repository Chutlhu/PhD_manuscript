\chapter{Conclusion}\label{ch:conclusion}
\openepigraph{But at the laste, every thing hath ende}{Geoffrey Chaucer}
Since the development of the DES and AES, our understanding of secure designs for encryption schemes has greatly evolved.
In particular in the area of symmetric cryptography, we are today, after more than 40 years of research, able to design very efficient ciphers, which we firmly believe to be secure -- with the \AES/ being the prime example withstanding 20 years of cryptanalysis.
Our progress pushed efficiency bounds further and further, especially within the trend of lightweight cryptography.

However the time may has come where we should shift our focus to improving security arguments for new designs -- because the improvement since the development of bounds for differential and linear cryptanalysis seems marginal.
We see this thesis, specifically the first part on security arguments, as a step in this direction.
With our block cipher instances bison and wisent we are for the first time able to give precise bounds on the \emph{differential} instead of only on differential trails.
This initial result may lead to further investigation of alternative constructions for block ciphers.
An interesting question in this direction is if a construction can be found which exhibits similar good properties with respect to linear cryptanalysis.
A second worthwhile direction is the study of unbalanced Feistel networks which seem to be related to the \WSN/ construction.

Apart from our results on differential cryptanalysis, our study of the \ACT/ revealed a connection between differential-linear cryptanalysis and previously studied properties of Boolean functions.
In our opinion the most interesting observation, from a cryptanalytic perspective, is that the decryption function might be weaker than the encryption against differential-linear attacks.
This result implies future analysis has to be extended in this direction.
From a more theoretical point of view, it is interesting that vectorial Boolean functions exhibit a lower bound for the absolute indicator, while for Boolean functions it seems to be a hard problem finding such a lower bound.
Overall, our results on this new connection contribute to a further understanding of differential-linear cryptanalysis.

In the second part of the thesis, we concentrated on automated tools for the design and analysis of block ciphers.
Our main result here was the conceptual simple algorithm for propagating subspaces through an iterative round function.
Despite the underlying simple idea, this algorithm turns out to be useful not only for one application.
For its original purpose, we use \textsc{Compute Trail} to algorithmically bound the longest subspace trail through an \SPN/ cipher and thus construct an algorithmic security argument against this recent type of attack.

However, besides the study of single attacks, a more principle task is to extend a distinguishing attack into a key recovery.
Especially when such an extension is possible over some rounds, it might make the difference between a cipher with a thin security margin and a broken one.
Thus, while being a very important part of cryptanalysis, finding key recovery strategies remains a highly manual, and thus error prone, task.
As discussed in the last chapter, our subspace trail algorithm may be used in an automatisation approach for exactly this problem -- albeit working out the exact techniques for such an automated key recovery remains to be done.

Apart from these possibilities for automated tools discussed in this thesis, a different application are cryptanalysis techniques based on \MILPp/.
We only briefly mentioned \MILPp/ for bounding the number of active S-boxes.
However, they have by now a broad spectrum of use cases, \eg/ for finding differential or linear trails, for finding division properties or similar.
All these applications have the same basic process that needs the cipher under scrutiny and the analysis technique to be modeled as an instance of the specific programming style, \ie/ as a \MILP/.
The needed building blocks for these models are known for every typical part used in ciphers, still the cryptanalyst has to assemble the models manually.
Again this is a tedious and error prone task which could easily be automated.
The development of such a \MILP/ compiler (or similarly a SAT compiler for constrained programming models) quite likely requires techniques from programming languages and compiler theory.
It seems to be an interesting problem to work on.

Finding the best representation of a cipher for these models (both for \MILPp/ and SAT) is another problem which yet remains unsolved.
This occurs especially when modeling the nonlinear S-boxes, for which different approaches exist: broadly speaking one could model the S-box in full detail, or try to pre-optimise the model on a varying level.
Similar to the XOR count optimisations it is then unclear, how much pre-optimisation helps in the end and what level of optimisation restricts the solver too much for its own optimisation strategies.
