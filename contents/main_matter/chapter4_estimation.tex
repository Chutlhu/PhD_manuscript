\chapter{Acoustic Echo Estimation}\label{chap:estimation}
\openepigraph{Signal, a function that conveys information about a phenomenon.
$[\dots]$ Consider an acoustic wave, which can convey acoustic or music information.}{R. Priemer, \textit{Introductory Signal Processing}}
\vspace{-2.5em}
\newthought{Synopsis} Let us now move from the physics to digital signal processing.
At first this chapter formalized fundamental concepts of audio signal processing such as signal, mixtures and noise~\cref{sec:processing:model} in the time domain.
In~\cref{sec:processing:domains} we will presents the signal representation that we will use throughout the entire thesis: the STFT domain.
Finally, after assuming the narrowband approximation, in~\cref{sec:processing:rirmodels} some important models for the \RIR/ are described.

Giving the frequency domain model of the \RIR/ defined in the previous chapters,
\begin{equation}
    \hat{h}[k] = \sum_{r=0}^K
\end{equation}
the \AER/ problem consists in estimating the echo timings $\kbrace{\tau_r}$ and attenuations $\kbrace{\alpha_r}$.
The term \AER/ is not typical in the audio signal processing community and it can be seen as instance of
channel estimation problem or time of arrival problems.

\section{as (sparse) \RIR/ estimation}
\begin{itemize}
    \item[def] estimation of the whole channel acoustic channel
    \item[methods] Signal know \vs/ unknown. statistical methods \vs/ blind method
\end{itemize}

\subsection{Other echo-related parameters estimation}
\newthought{\TDOA/ estimation}
\begin{itemize}
    \item[def] estimation of the the difference of the direct path
    \item[methods] Cross-correlation
\end{itemize}

\newthought{Echo Density Estimation}

\newthought{$\RT$ and $\DRR$ estimation}

\section{Acoustic Echo Estimation is}

\begin{itemize}
    \item Acoustic Echo Retrieval definition
    \item Acoustic Echo Retrieval scope and placement in the signal processing pipeline
    \item Acoustic Echo Retrieval characteristic
\end{itemize}

\section{Echoes in the Time, Frequency and Cepstral domains}
\begin{itemize}
    \item Time domain processing
    \item Frequency domain processing
    \item Correlation processing
    \item Cepstral processing
\end{itemize}


\section{Related Works}
\subsection{Active vs. Passive echoes estimation}

\subsection{Knowledge-based vs. Data-driven}
\begin{itemize}
    \item Knowledge-driven (Physic-driven)
    \begin{itemize}
        \item Channel (RIR) estimation and Echoes pruning - Crooco and Dokmanic
        \item TDOA estimation (multipath) - Benesty
        \item Spikes Retrieval - Condat
    \end{itemize}
    \item Data-driven
    \begin{itemize}
        \item GLLiM
        \item Deep Leaning echo estimation
    \end{itemize}
\end{itemize}

\subsection{end-2-end vs 2-steps approaches}

AER\\
eRTF + AER

Pruning methods

\section{Related Works}

\subsection{AER as a RIR Estimation problem}
\todo{summarize Crocco's presentation}

TX signal: known vs. not known
\\TX signal not known: statistical methods and blind methods

\subsection{AER as a Spike Estimation problem}


\subsection{Virtually-supervised and Data Augmentation}


\section{Data and Metrics}
\subsection{Spike-based metrics}