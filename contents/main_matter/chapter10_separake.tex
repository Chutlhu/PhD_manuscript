\chapter{\separake: Sound Source Separation with Echoes}\label{chap:separake}

It is commonly believed that multipath hurts various audio processing algorithms.
At odds with this belief, we show that multipath in fact helps sound source separation,
even with very simple propagation models. Unlike most existing methods,
we neither ignore the room impulse responses, nor we attempt to estimate them fully.
We rather assume to know the positions of a few virtual microphones generated by
echoes and we show how this gives us enough spatial diversity to get a performance
boost over the anechoic case. We show improvements for two standard algorithms---one that
uses only magnitudes of the transfer functions, and one that also uses the phases.
Concretely, we show that multi-channel non-negative matrix factorization aided with
a small number of echoes beats the vanilla variant of the same algorithm,
and that with magnitude information only, echoes enable separation where it was previously impossible.

\section{Introduction}

\subsection{Literature review: an acoustic perspective}
Bibliography with respect to sound propagation

\subsection{Literature review an algorithmic perspective}
Bibliography with respect to learning and knowledge approaches

\section{Background in SSS}

\newthoughtpar{SSS by NMF}

\newthoughtpar{NMF using Multiplicative-Updates (MU-NMF)}

\newthoughtpar{NMF using Expectation-Maximization (EM-NMF)}

\section{\separake: SSS with echoes}

\section{Experimental evaluation}

\section{Conclusion}