\chapter{Overture}\label{chap:intro}
\openepigraph{Only echoes answer me.}{Anton Chekhov, Swan Song}%
\openepigraph{\textsc{Écho.} Citer ceux du Panthéon et du pont de Neuilly. }{Gustave Flaubert, Dictionnaire des idées reçues}
\openepigraph{\textsc{``Echoes''} shows the direction that we’re moving in. }{David Gilmour, about the making of ``The Dark Side Of The Moon''}

\vspace{-2.5em}

\def\MyText{\textsc{Echooooooooooes}}
% \newcommand\shadetext[2][]{%
%   \setbox0=\hbox{{\special{pdf:literal 7 Tr }#2}}%
%   \tikz[baseline=0]\path [#1] \pgfextra{\rlap{\copy0}} (0,-\dp0) rectangle (\wd0,\ht0);%
% }
\newcommand\mytext[2][black]{\scalebox{#2}{\textcolor{#1}{\MyText}}}

% \newthought{Picture yourself} standing in room you've never been before, blindfold.
% Clap your hands and listen. Can you tell the size of the room?


% This thesis is about taking inspiration from people, bats, Daredevil and Pink Floyd.

% \stackinset{c}{4ex}{c}{ 1.4ex}{\mytext[black!70]{.8}}{%
% \stackinset{c}{2ex}{c}{ 3.2ex}{\mytext[black!25]{.7}}{%
% \stackinset{c}{3ex}{c}{-1.3ex}{\mytext[black!35]{.9}}{%
% \mytext{1}%
% }}} and how these properties can be exploited for analyzing auditory scenes.


% % \\We live immersed in a complex acoustical world, where every concrete thing sounds and resounds.
% % Sounds are produced by sources and interacts with the surrounding space while reaching our hears or being recorded by microphones.
% % \\\textsc{sounds convey and contain information}: the semantics



% Acoustic echoes are a natural phenomena created by a particular interaction of the sound with the environment.

% Animals and humans have a remarkable ability to listen to the acoustic response of their environment.
% Also known as \emph{echolocation} or \emph{bio-sonar}, it is used consciously and unconsciously to retrieve
% information about the environment and objects using sound waves.
% Two (of the most) striking examples are bats and whales which use it as navigation and foraging mechanisms.


% \marginpar{%
%     \begin{itemize}
%         \item[\faYoutube] \href{https://www.youtube.com/watch?v=lLUcOFwZvyY&t=22s}{Testing The World's Longest Echo}
%         \item[\faYoutube] \href{https://www.youtube.com/watch?v=px3oVGXr4mo}{SKUNK BEAR : What Does Sound Look Like?}
%         \item[\faYoutube] \href{https://www.youtube.com/watch?v=ZwgovJSQ5UM}{ARTE : La Magie Du Son}
%         \item[\faYoutube] \href{https://www.youtube.com/watch?v=uH0aihGWB8U&t=631s}{Daniel Kish: How I use sonar to navigate the world}
%     \end{itemize}

% }

% Definitions of: In this thesis, an \textit{auditory scene} consists in \textit{sound sources}, \textit{microphones} deployed in a room.
\newthought{ASD}\lipsum[1]

\section{The Problem}\label{sec:intro:problem}
In the context of audio signal processing, algorithms can be grouped according to how they deal with sound propagation.

\section{My Thesis}
The goal of this dissertation is to improve the above state of affairs along two axes:
First, by deepening our understanding of sophisticated scalable algorithms, isolating their essence and thus reducing the barrier to building new ones. Second, by developing new ways ofexpressing scalable algorithms that are abstract, declarative, and user-extensible.
To that end, the dissertation demonstrates two claims:
\begin{itemize}
    \item
    Scalable algorithms can be understood through linked protocols govern- ing each part of their state, which enables verification that is local in space, time, and thread execution.
    \item
    Scalable algorithms can be expressed through a mixture of shared-state and message-passing combinators, which enables extension by clients without imposing prohibitive overhead.
\end{itemize}
We elaborate on each claim in turn.

\subsection{Audio Signal Processing}
\begin{itemize}
    \item Motivation
    \item Definitions, Function, Characteristics
    \item Current challenges
\end{itemize}
\newthought{Inverse Problem}
Starting with the effects to discover the causes has concerned physicists for centuries.

While in many ways, mixtures are not different to any other audio signal, two research questions stand out prominently: • Can we obtain the sources sj from the mixture x? • Can we find the number of sources J from x? These two questions are addressed in the scientific fields of sound
source separation and source count estimation


Inverse problems appear when we want to see or examine something that we cannot access directly. What we have is an indirect measurement that contains hidden information.

An inverse problem is always a counterpart of a direct problem, as shown in the schematic diagram below. The direct problem is going from object to data, and the inverse problem is about finding the object back from the data.

The assumed few thousand taps. This model was very popular in the early stages of research [48]–[55]. Recently, interest has revived with sparse penalties which account for prior knowledge about the physical properties of AIRs, namely the facts that power concentrates in the direct path and the first early echoes [56]– [60] and that the time envelope decays exponentially [61], but these penalties have not yet been used in a BSS context.

\subsection{Echo-aware Processing}
In the everyday context, when a sound reflection is perceived distinctly is referred to as \textit{echo}.
While phenomenon can be observed clearly in outdoors environment, such in the mountains or within huge buildings,
in closed rooms it is less noticeable. In fact, echoes are usually masked by a general reverberation of the room.

\begin{itemize}
    \item Motivation
    \item Definitions, Function, Characteristics
    \item Current challenges
\end{itemize}

Auralization is the process of rendering audible, by physical or mathematical modelling, the sound field
of a source in a space, in such a way as to simulate the binaural listening experience at a given position in the modelled space

\section{Audio Inverse Problems}\label{sec:processing:inverse}
\cite{kitic2015cosparse}
\openepigraph{Their generality is of such a wide scope that onemayeven argue that solving inverse problems is what signal processing is all about}{Srdan Kiti\'c, \textit{Cosparse regularization of physics-driven inverse problems}}
\openepigraph{everything is an optimization problem}{\citeonly{watson2001nonlinear}}
\marginpar{
    \footnotesize
    One can see the paralelism with the engineering concepts: analysis and sythesis.
}
In~\cref{sec:intro:problem} we have informally defined \textit{inverse problems}, with an emphasis on inverse problems in signal processing.
An inverse problem is a type of a mathematical problem where we start with the observations and we want to estimate model parameters that produced them.
\\Inverse problems pervades all the field of science and engineering:
source localization~\cite{},
image processing~\cite{},
acoustic imaging and tomography~\cite{},
\marginpar{
    \footnotesize
    A historical example are the calculation of the Earth circunference by Eratosthenes in III century b.c.\\
    and the calculations of Adams and Le Verrier which led to the discovery of Neptune from the perturbed trajectory of Uranus.
}

A inverse problems is defined as the counterpart of a \textit{forward}\sidenote{often referred to as \textit{direct}} problem.
Without falling in and deep mathematical formalism and taxonomies which can be found in \citeonly{bal2012introduction},
we will simply consider the following informal definition:
\begin{center}
    \textit{\emph{Forward problem} starts from known input, while \emph{inverse problem} starts from known output~\cite{santamarina2005discrete}.}
\end{center}
Both these problems focus on an operation relating maps objects of interest, called \textit{parameters} or \textit{variables},
to information collected about these objects, called \textit{measurements}, \textit{data} or \textit{observation}.

For instance, in our context, the direct problem may be the estimation of the \RIR/(s) starting from the known room parameters,
and, the related inverse problem would be the estimation of such room properties from the observation of the \RIR/(s).

Formally, a forward problem is defined through a mathematical model, described by a \textit{operation} $\scrM(\cdot)$
mapping \textit{parameters} $x \in \scrX$ to the \textit{observation} (or measurement) $y \in \scrY$:
\begin{equation}\label{eq:processing:model}
    y = \scrM(x)
    .
\end{equation}
Then, the inverse problem defines a method $\kinv{\scrM}$ that ``reverts'' $\scrM$ in order to recover (estimate) $x$ form the observation of $y$.
% The operator $\scrM$ describes our best effort to construct a \textit{model} for the available data $y$.
% The choice of $\scrX$ describes our best effort to characterize the space where we believe the parameters belong.

As discussed in~\cite{bal2012introduction}, \textit{solving} the inverse problem consists in finding point(s) $x \in \scrX$ from (knowledge of) data $y \in \scrY$
such that~\cref{eq:processing:model} or an approximation of~\cref{eq:processing:model} holds.
Under this light, the operator $\scrM$ ant the choice of $\scrX$ describes our best effort to construct a \textit{model} for the data $y$ and
the space where the parameters $x$ belong, respectively.
\marginpar{
    \footnotesize
    one can already see the paralelism the the definition of the mixing process defined in~\cref{sec:intro:problem}
}

\textsc{For instance, in Case of} \textit{linear} inverse problem, and for $\scrY$ and $\scrX$ being vector spaces of dimensions $M$ and $N$ respectively,
then the forward map can be written as a linear system:
\begin{equation}\label{eq:processing:linear_forward}
    \bfy = \bfM \bfx
\end{equation}
where $\bfM$ being a matrix, namely the operator $\scrM$ becomes a matrix multiplication by $M$.
It follows that the inverse map associated to~\cref{eq:processing:linear_forward} is the application of the inverse matrix $\kinv{M}$.
% While solving a direct problem the an operator needs to be found, in solving the inverse one either the operator is known and needs
% to be $reverts$t

Typically, forward problems are considered somehow the ``easier''.
In fact, even in the observation model $\scrM$ is known perfectly, it is not always possible to find its counterpart.
This because of
\begin{itemize}
    \item presence of \textit{noise} in the measurement which are not always additive and statistically independent \wrt/ $x$.
    \item the problem is \textit{well-posed} and \textit{well-conditioned}, namely $\scrM$ needs be injective and stable.
    In other words, some information is recoverable, other is completely lost, other highly sensible to noise
    \sidenote{
        \textbf{injective} ensure the uniqueness of the solution, while \textbf{stability}
        ensure a continuity on the data.
        These are known as the Hadamard's \textit{solvability conditions}.
    }.
\end{itemize}

As we could images, many interesting and fundamental inverse problem are
\textit{ill-posed} or \textit{ill-conditioned} in general, even in the following ``simple'' ones~\cite{kitic2015cosparse}:
The solution to the deconvolution problem, where the direct inversion of the transfer function results in instabilities
at high frequency; and the solution a linear system $\bfy = \bfM \bfx$ where $\bfM$ is invertible
may lead to erroneous results and numerical instabilities.

Therefore, sometimes ones have to settle for restring the set of solution $\scrC \subset \scrX$,
where $\scrM$ is stable and injective\sidenote{This framework was originally proposed by Tikhonov.}.
Promoting solution $x \in \scrC$ is can be achieved through \textit{model priors}, namely prior knowledge about solution, which can
be classified in the following methodologies:
the usage of \textit{geometric constraints} that deterministically define the solutions; the imposition of \textit{penalization}
which ``promotes'' solution of a certain shape (\eg/ \textit{sparse}
\sidenote{\textbf{sparsity} is a fundamental concept of this thesis, better discussed in~\cref{pt:estimation}
} or \textit{smoothness});
and casting the problem in a \textit{bayesian framework} which versatilely incorporate prior and posterior density function describing the data.

\subsection{General Processing Scheme}
Digital signal processing (DSP) is the process of analyzing and modifying a signal to optimize or improve its efficiency or performance. It involves applying various mathematical and computational algorithms to analog and digital signals to produce a signal that's of higher quality than the original signal.
It is traditional in engineering to represent complex systems as a collection of simpler subsystems, with well-defined tasks, interacting with each other.
In signal processing, these subsystems roughly fall into four categories: \textit{representation}, \textit{enhancement}, \textit{estimation}, and \textit{adaptive processing}.
Many problems can be decomposed into blocks that belong to one of these categories.

\begin{description}
    \item[Representation] Objects can be represent (described) in many different way.
    Through different representations, some object \textit{information} becomes more relevant and suitable for certain tasks
    than other.
    \\Representation can be lossy or lossless, and are generally implemented through (non)linear mapping, such as change of basis or feature.
    The most famous representation is the Fourier basis.
    \\Depending on the task the representation may be invertible.
    The process of changing representation is often called: Analysis and Synthesis

    \item[Enhancement] Measurement are affected by noise and interferences which corrupt and hide relevant information, making inverse problems ill-posed and ill-conditioned.
    Therefore, signal enhancement, that is removing noise, is a necessary step.
    \\Enhancement constitute a huge dome of methods: form simple denoising by averaging of repeated measurement to
    spectral subtraction to source separation with neural network.

    \item[Estimation] Often we wish to estimate some key properties of the target signal which may be used as inputs to a different algorithm.

    \item[Adaptive processing] deals with adaptive algorithms and filters controlled by variable parameters.
    A common means to adjust those parameters according to an optimization algorithm which rely on statistical properties of the signal of interest.
    They often implement a kind of online optimization where an objective function is being minimized.
    When new data is observed, its discrepancy with the current estimate is used to produce a new estimate in a way that reduces the objective.
\end{description}

Let us give two example of practical systems that will be recurrent thought out the entire thesis.

\subsection{Selected Audio Inverse Problems}
Here follow some famous problems in the field of audio signal processing with application to speech, music and environmental audio.
Given the mixing process defined in~\cref{sec:processing:model},

% \begin{description}
%     \item[sound source separation and enhancements] as the problem of retrieving a (set of) source signal from a mixture.
%     \item[sound source localization] estimation of source location from the observation of the sound production.
%     This has sense as long as the impulse response convey space properties.
%     \item[microphones calibration] estimation of the microphone placement.
%     \item[\RIR/ estimation] estimation of the filters.
%     Blind Channel Estimation or System Identification.
%     \item[Acoustic Echoes Estimation] estimation of the filters
%     \item[dereverberation] estimation of the filters
%     \item[room geometry estimation] estimation of the room
%     \item[automatich speech recognition]
% \end{description}

\begin{tabular}{p{0.33\linewidth}|p{0.66\linewidth}}
    \toprule
    Inverse Problem & \textit{Can we estimate the...} \\
    \hline
    Audio Source Separation  & the signal of the sources $s_{j}$ from the mixture $\boldsymbol{x}$? \\

    Sound Source Localization & the position $\mathbf{s}_{j} \ =\ [ x_{s_{j}} ,\ y_{s_{j}} ,\ z_{s_{j}}]$  of the source $s_{j}$ from the mixture $\boldsymbol{x}$$ $? \\

    Microphone (Array) Calibration & the position of the microphone (array) position $\mathbf{x}$ from the mixture $\boldsymbol{x}$? \\

    \RIR/ Estimation & the filter between the sources $\boldsymbol{s}_{j}$ and the mixture $\boldsymbol{x}$ from $\boldsymbol{x}$? \\

    Room Geometry Estimation & the shape of the room in which the mixture $\boldsymbol{x}$ recoding source $s_{j}$? \\
    \bottomrule
\end{tabular}

\openepigraph{Everything is connected}{Douglas Adams, \textit{Dirk Gently's Holistic Detective Agency}}
\newthought{Depending on the scenario}, all these problems exhibits strong inter-connections,
namely the solution of one may be (dependent on) the solution of another.
Therefore, exploiting expertise and knowledge,
interconnect and hierarchical approaches may be built\sidenote{Machine Learing allows now for end2end approaches}:
for instance, many spatial filtering techniques used for \SE/ rely on \SSL/ blocks;
and in order to achieves \RooGE/, \AER/ must be done.


\section{My Thesis}
\subsection{Hunting Acoustic Echoes}
\subsection{Echo-aware Auditory Scene Analysis}


\section{Organization and Contributions}
\newthoughtpar{Room Acoustic meets Signal Processing}
\subparagraph{\cref{ch:acoustics}}
\subparagraph{\cref{ch:processing}}
\subparagraph{\cref{ch:evaluation}}

\newthought{Hunting Acoustic Echoes}
\subparagraph{\cref{ch:estimation}}
\subparagraph{\cref{ch:lantern}}
\subparagraph{\cref{ch:blaster}}
\subparagraph{\cref{ch:blasterr}}


\newthought{Echo-aware Auditory Scene Analysis}
\subparagraph{\cref{ch:application}}
\subparagraph{\cref{ch:separake}}
\subparagraph{\cref{ch:mirage}}
\subparagraph{\cref{ch:brioche}}


Finally, the dissertation concludes with Chapter X, which summarizes
the contributions and raises several additional research questions


\section{This Thesis: Don't Panic!}
The reader will have already noticed that a large margin is left free on the right side of each page of the manuscript.
We will use it to insert comments, historical notes as well as figures and tables to complete the subject.
This graphic charter is inspired by the work of Tufte (2001) and produced using the latex tufte-latex class.
We emphasize that the presence of the clickable GitHub logo in the margin indicates the online availability of the codes.

\newthought{Quick vademecum} for the readers:
\begin{itemize}
    \item Bibliographic references are denoted as \cite{kuttruff2016room}.
    \item Figures, Tables and other floating objects as well as equations are numbered within the chapter number.
    \item Equations are referred as~\cref{eq:acoustics:green_definition}
    \item The main matter of the Thesis’s manuscript starts at page 1, until page 103.
    \item The back matter covers the list of the candidate’s publications and the bibiographic references cited along the text.
    \item Small notes on the margin might be used to easily navigate through the Example of margin note manuscript. They are meant to summarize paragraphs/blocks of text.
    \item The end of the chapter is shown by the following sign between horizontal rules.
\end{itemize}

\newthought{The golden ratio of the thesis}
\begin{itemize}
    \item at most 3 level of sub-headings: section, subsection and new-thought
    \item usage of dichotomies are preferred
    \item each paragraph is introduced briefly at the end of the previous one
    \item definition are provided with stacco
    \item Not important figures: without numbering
\end{itemize}