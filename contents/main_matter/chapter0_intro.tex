\chapter{Prologue}\label{chap:intro}

\openepigraph{Only echoes answer me.}{Anton Chekhov, Swan Song}
\openepigraph{\textsc{Écho.} Citer ceux du Panthéon et du pont de Neuilly. }{Gustave Flaubert, Dictionnaire des idées reçues}

\section{Preamble}
Animals and humans have a remarkable ability to listen to the acoustic response of their environment.
Also known as \emph{echolocation} or \emph{bio-sonar}, it is used consciously and unconsciously to retrieve
information about the environment and objects using sound waves.
Two (of the most) striking examples are bats and whales which use it as navigation and foraging mechanisms.


\marginpar{%
    \begin{itemize}
        \item[\faYoutube] \href{https://www.youtube.com/watch?v=lLUcOFwZvyY&t=22s}{Testing The World's Longest Echo}
        \item[\faYoutube] \href{https://www.youtube.com/watch?v=px3oVGXr4mo}{What Does Sound Look Like? | SKUNK BEAR}
        \item[\faYoutube] arte
    \end{itemize}

}

\section{The Problem}
\subsection{Audio Signal Processing}
\begin{itemize}
    \item Motivation
    \item Definitions, Function, Characteristics
    \item Current challenges
\end{itemize}
\subsection{Echo-aware Processing}
In the everyday context, when a sound reflection is perceived distinctly is referred to as \textit{echo}.
While phenomenon can be observed clearly in outdoors environment, such in the mountains or within huge buildings,
in closed rooms it is less noticeable. In fact, echoes are usually masked by a general reverberation of the room.

\begin{itemize}
    \item Motivation
    \item Definitions, Function, Characteristics
    \item Current challenges
\end{itemize}

Auralization is the process of rendering audible, by physical or mathematical modelling, the sound field
of a source in a space, in such a way as to simulate the binaural listening experience at a given position in the modelled space

\section{My Thesis}
\subsection{Hunting Acoustic Echoes}
\subsection{Echo-aware Auditory Scene Analysis}


\section{Organization and Contributions}
\newthoughtpar{Room Acoustic meets Signal Processing}
\subparagraph{\cref{chap:acoustics}}\blindtext
\subparagraph{\cref{chap:processing}}\blindtext
\subparagraph{\cref{chap:evaluation}}\blindtext

\newthought{Hunting Acoustic Echoes}
\subparagraph{\cref{chap:estimation}}\blindtext
\subparagraph{\cref{chap:lantern}}\blindtext
\subparagraph{\cref{chap:blaster}}\blindtext
\subparagraph{\cref{chap:blasterr}}\blindtext


\newthought{Echo-aware Auditory Scene Analysis}
\subparagraph{\cref{chap:application}}\blindtext
\subparagraph{\cref{chap:separake}}\blindtext
\subparagraph{\cref{chap:mirage}}\blindtext
\subparagraph{\cref{chap:brioche}}\blindtext


Finally, the dissertation concludes with Chapter X, which summarizes
the contributions and raises several additional research questions


\section{Don't Panic}
How to read a This (Tufte) Thesis