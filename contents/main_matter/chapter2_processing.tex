\chapter{Audio Inverse Problem}\label{chap:processing}


\section{Signal Model}
\subsection{Multichannel Mixing Process}

\subsection{Time-Frequency Analysis and Synthesis}

\subsection{Artificial Mixtures}

\subsection{Impulse Response Models}
\newthoughtpar{Acoustic and Room Impulse Response}

\newthoughtpar{Acoustic and Room Transfer Functions}

\newthoughtpar{Steering Vectors}

\newthoughtpar{Relative Transfer Functions}

\newthoughtpar{Full-Rank Covariance Models}

%%%%%%%%%%%%%%%%%%%%%%%%%%%%%%%%
\section{Audio Inverse Problems}
\newthoughtpar{Forward vs Inverse Problem}

\subsection{General Processing Scheme}
General Processing Pipeline

\subsection{Some Audio Inverse Problems}
\begin{enumerate}
    \item sound source separation and enhancements
    \item sound source localization
    \item microphones calibration
    \item channel estimation
    \item room geometry estimation
    \item acoustic echo estimation
\end{enumerate}
\newthoughtpar{Evaluation}

%%%%%%%%%%%%%%%%%%%%%%%%%%%%%%%%
\section{Taxonomy through dichotomies}
%% acoustics %%
\newthoughtpar{Single-Channel vs. Multichannel}
\newthoughtpar{Point vs. Diffuse Sources}
\newthoughtpar{Directonal vs. Onmidirectional Recordings}
\newthoughtpar{Diffuse vs. Measurements Noise}
%% processing %%
\newthoughtpar{Natural vs. Artificial Mixtures}
%% pipeline %%
\newthoughtpar{Problem vs. Model}
\newthoughtpar{Synthesis vs. Abstaction}
%% problems %%
\newthoughtpar{Separation vs. Enhancement}
%% models %%
\newthoughtpar{end2end vs. 2step}
end2end: from data to (feature to) target
\\2-step: (from data to features) + features to target

\newthoughtpar{Knowledge-based vs. Learning-based}
Knowledge-based: specialized signal processing and mathematical algorithms informed by knowledge;
Learning-based: machine learning usually trained in supervised fashion.

\newthoughtpar{Supervised vs. Unsupervised}
\newthoughtpar{Machine Learning vs. Deep Learning}