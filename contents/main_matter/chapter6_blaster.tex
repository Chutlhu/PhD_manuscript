\chapter{Knowledge-driven Acoustic Echo Retrieval \& \blaster}\label{chap:blaster}

\marginpar{%
\footnotesize
\textbf{Keywords:} Blind Channel Identification, Super Resolution, Sparsity, Acoustic Impulse Response.
}
\newthought{Synopsis} Acoustic echoes retrieval is a research topic that is gaining importance in many speech and audio signal processing applications
such as speech enhancement, source separation, dereverberation and room geometry estimation.
This work proposes a novel approach to blindly retrieve the \textit{off-grid} timing of early acoustic echoes %\textit{i.e.},
from a stereophonic recording of an unknown sound source such as speech. It builds on the recent framework of continuous dictionaries.
In contrast with existing methods, the proposed approach does not rely on parameter tuning nor peak picking techniques by working directly in the parameter space of interest.
The accuracy and robustness of the method are assessed on challenging simulated setups with varying noise and reverberation levels and are compared to two state-of-the-art methods.

\section{Introduction}
\label{sec:blaster:intro}
%Striking introducing sentence
In room acoustics and audio signal processing, the temporal structure of the room impulse response (RIR) plays a central role.
% First echo is noise
It is the result of multiple (indirect) sound propagation paths due to specular and diffuse reflections on the room's surfaces, leading to reverberation~\cite{Wang2011}.
In such conditions, the perceived sound quality is often considered degraded and it is common to observe a detrimental decrease of performance as reverberation increases for applications such as speech recognition~\cite{Yoshoka2012} or music information retrieval~\cite{Barthet2010}.
% or music virtual reality~\cite{DeMan2017}. %to observe a detrimental decrease of performances with reverberation. as well as music virtual reality~\cite{DeMan2017}.
%\textcolor{red}{Add something about amazon echo, google assistant (analysis) and virtual reality (rendering/synthesis)}

% Second echo is information
On the other hand, RIRs contain very rich geometrical information about the acoustic scene.
%which is independent from the source signal itself.
%In contrast with well-consolidated methods,
Recent \textit{echo-aware} works have shown that the knowledge of the timing of early reflections may boost performance in many audio signal processing applications,
from dereverberation~\cite{Wu2006,Lin2008} to sound localization~\cite{Ribeiro2010,DiCarlo2019} and separation~\cite{Dokmanic2015a, Scheibler2017}.
Moreover, it allows joint estimation of the receivers' positions~\cite{Salvati2016}, the reflective surfaces~\cite{Antonacci2012} and consequently the geometry of the room~\cite{Dokmanic2013, Crocco2017}.
% or acoustic impedance of surfaces~\cite{Antonello2014, Bertin2016}.
% beamforming \cite{Dockmanic2015},
% such as for speech enhancement \cite{Wu2006}, source localization \cite{Ribeiro2010}, source separation \cite{Scheibler2017} and dereverberation \cite{Lin2009}
% which are common pre-processing steps for many applications \cite{Gannot2017}

Acoustic echo retrieval (AER) consists in estimating the properties of the early (strong) acoustic reflections only in
multi-path environments~\cite{Tukuljac2018}, sometimes referred to as time delay estimation~\cite{Chen2006a}.
To achieve this, several methods rely on a known source signal~\cite{park2017compressive,jensen2019method}.
% Adding unknown adds a line :-(
% yes, but it is really important
% Got it. New mission: win space somewhere else in the paragraph?
% no problem, I will pass throughthhtghththghth this
% Seems legit
In contrast, when multiple receivers attend an unknown single source, AER can be seen as an instance of Single Input Multiple Output (Blind) Channel Estimation (SIMO-BCE) problem.
%or \textit{system identification}, \textit{i.e.} estimating the filters entailing an unknown input observed output of a system.
%In a more challenging setting, when no prior information about the source signal are available, the problem is referred to as SIMO \textit{blind channel estimation} (SIMO-BCE) \cite{Lin2007}.
A common approach for solving AER in the context of SIMO-BCE is to first blindly estimate a discrete version of the acoustic channels using the so-called cross-relation identity~\cite{Xu1995, Crocco2016}.
The location of the echoes are then chosen among the strongest peaks with ad-hoc peak-picking techniques.
%Such methods are generally \emph{on-grid} in the sense that the estimation relies on a fixed grid of points and \textit{a priori} chosen filter lengths.
However, in practice, the true timings of echoes rarely match the sampling grid, thus leading to pathological issues called basis-mismatch in the field of compressed sensing.
To circumvent this issue, the authors of~\cite{Tukuljac2018} proposed to leverage the framework of finite-rate-of-innovation sampling to make one step towards off-grid approaches.
Despite promising results in the absence of noise and with synthetic data, the quality of the estimation highly relies on an initialization point.

Of particular interest in this paper is the recently proposed framework of continuous dictionaries (CD)~\cite{Carlos2014}.
By formulating an inverse problem as the recovery of a discrete measure over some parameter space, CD has allowed to overcome imaging device limitations in many applications such as super-resolution~\cite{Carlos2014} or PALM/STORM imaging~\cite{denoyelle2019}.
In this work, we formulate the problem of stereo AER within the framework of continuous dictionaries.
The resulting optimization problem is convex and thus not prone to spurious minimizers.
The proposed method is coined \emph{Blind And Sparse Technique for Echo Retrieval} (\blaster) and requires no parameter tuning.
The method is compared to state-of-the art on-grid approaches under various noise and reverberation levels using simulated data.
While comparable or slightly worse recovery rates are observed for the task of recovering 7 echoes or more, better results are obtained for fewer echoes and the off-grid nature of the approach yields generally smaller estimation errors.


\section{Background in Acoustic Echo Estimation}
\label{sec:blaster:background}

\subsection{Signal and measurement model}
Consider the common setup where a band-limited and square-integrable source signal $\contSource$ is emitted.
Due to the geometry of the room, the latter signal is both reflected (several times) and attenuated before reaching a set of two microphones.
The recorded signal at microphone $i\in\{1,2\}$ reads
\begin{equation}
    \label{eq:recordedSignal}
    \contRecordedSignal_i = \contSource \ast \contFilter_i^\star + \contNoise_i
\end{equation}
where $\ast$ denotes the (continuous) convolution operator, $\contNoise_i$ models some additive noise in the measurement process and $\contFilter_i^\star$ denotes the room impulse response (RIR).
In the remainder of this paper, the  superscript $\star$ refers to the ground truth.
In AER, we are interested in RIRs that are streams of Diracs, \textit{i.e.},
\begin{equation}
    \label{eq:def_filter_star}
    \contFilter_i^\star(t) = \sum_{r=0}^{R_i-1} c_{i,r} \delta(t - \tau_{i,r})
\end{equation}
where $R_i$ is the (unknown) number of echoes, $\kfamily{\tau_{i,r}}{r=0}^{R_i-1}$ models the echoes' delays, and $\kfamily{c_{i,r}}{r=0}^{R_i-1}$ are the corresponding non-negative attenuations.
Note that $r=0$ defines the direct propagation path.
%
In the noiseless case, that is when $\contNoise_i=0$ for $i\in\{1,2\}$, we have the identity
\begin{equation} \label{eq:cross-relation}
    \contRecordedSignal_1 \ast \contFilter_2^\star = \contRecordedSignal_2 \ast \contFilter_1^\star
\end{equation}
by commutativity of the convolution operator.
This result is dubbed cross-relation identity in the channel identification literature \cite{Xu1995}.
Hence, one can expect to recover the two filters by solving an optimization problem involving~\eqref{eq:cross-relation}.
%the difference between the two terms in~\eqref{eq:cross-relation}.

However, in practice, only sampled versions of the two recorded signals are available.
More precisely, we consider a  measurement model where the incoming signal undergoes a (ideal) low-pass filter $\idealLowPassFilter$ with frequency support $\kintervcc{\sfrac{-\Fs}{2}}{\sfrac{\Fs}{2}}$ before being regularly sampled at the rate  $\Fs$.
We denote $\disRecordedSignal_1,\disRecordedSignal_2\in\kR^{2N}$ the two vectors of $2N$ (consecutive) samples and $i\in\{1, 2\}$ by
\begin{equation}
    \label{eq:measurement-process}
    \disRecordedSignal_i[n] =
    \kparen{\idealLowPassFilter \ast \contRecordedSignal}\kparen{\frac{n}{\Fs}}
    \qquad
    \forall n \in\{0, \dots, 2N-1\}
    .
\end{equation}
%Let $\disRecordedSignal_1,\disRecordedSignal_2\in\kR^N$ denote the two vectors of $N$ consecutive samples that are such that $\disRecordedSignal_i[\ell] = \contRecordedSignal_i(\sfrac{\ell}{F_s})$ for $\ell\in\{1...N\}$, $i\in\{1, 2\}$ and  where $F_s$ is the sampling frequency.



%We denote $\disRecordedSignal\in\kR^N$ the $N$ consecutive samples, \textit{i.e.}, such that $\disRecordedSignal[\ell] = \contRecordedSignal(\sfrac{\ell}{F_s})$ where $F_s$ is the sampling frequency.
% These $N$ consecutive samples will be denoted  $\disRecordedSignal\in\kR^N$ and satifies $\disRecordedSignal[\ell] = \contRecordedSignal(\sfrac{\ell}{F_s})$ $\forall\ell\in\{1...N\}$ where $F_s$ is the sampling frequency.

%the signals are notaccessible.  They are measured by sensors and discretized to be stored in a computer’s memory.


\subsection{Existing works}
Starting from the identity \eqref{eq:cross-relation}, the common SIMO BCE cross-relation framework aims to compute $\contFilter_1, \contFilter_2$ solving the following LASSO-type problem in the discrete-time domain:
\begin{multline}
    \label{eq:xrel_toepl}
    %\begin{split}
    \disFilterHat_1, \disFilterHat_2
    =
    \kargmin_{\disFilter_1, \disFilter_2}
    \;
    \kvvbar{
        \calT(\disRecordedSignal_1) \disFilter_2
        -
        \calT(\disRecordedSignal_2) \disFilter_1
    }_2^2
    +
    \lambda
    \kvvbar{
        \disFilter
    }_1
    \\
    %\quad
    \text{s.t.} \quad \disFilter[0] = 1
    \qquad
    %\qquad
    %\end{split}
\end{multline}
where $\disRecordedSignal_i$ and $\disFilter_i$ are the discrete, sampled version of $\contRecordedSignal_i, \contFilter_i$ respectively and $\disFilter = [\disFilter_1^\intercal, \disFilter_2^\intercal]$.
$\calT(\disRecordedSignal_i)$ is the $(2N+L-1) \times L$ Toeplitz matrix\footnote{The first row and column of $\calT(\disRecordedSignal_i)$ are respectively $[\disRecordedSignal_i[2N-n], 0,\dotsc,0]$ and $[\disRecordedSignal_i[2N-n], \disRecordedSignal_i[2N-n+1],\dotsc,\disRecordedSignal_i[n], 0,\dotsc, 0]^\intercal$.} associated to convolution where $2N$ and $L$ respectively denote  microphone and filter signal length.
The constraint $\disFilter[0]=1$ is called an anchor constraint.

The accuracy of estimated RIRs has been subsequently improved using a priori knowledge of the filters: in particular, the authors of~\cite{Lin2007} have proposed to use sparsity penalty and non-negativity constraints to increase robustness to noise as well as Bayesian-learning methods to automatically infer the value of $\lambda$ in~\cite{Lin2008}.
Even if sparsity and non-negativity could be seen as a strong assumption, works in speech enhancement~\cite{Ribeiro2010,Dokmanic2015a} and room geometry~\cite{Antonacci2012,Crocco2017} estimation have proven the effectiveness of this approach.
On a similar scheme, in~\cite{Kowalczyk2013},~\eqref{eq:xrel_toepl} is solved using an adaptive time-frequency-domain approach while~\cite{Aissa-El-Bey2008} proposes to use the $\ell_p$-norm instead of the $\ell_1$-norm.
A successful approach has been presented recently by Crocco \textit{et al.} in \cite{Crocco2016}, where the anchor constraint is replaced by an \textit{iterative weighted} $\ell_1$ equality constraint.
%, \corrCE{ \textit{i.e.}, such that $\langle{\bfh^{(t-1)},\bfh^{(t)}}\rangle=1$ at each iteration $t$ where $\bfh^{(0)}$ is the solution of~\eqref{eq:xrel_toepl}.}
%$\bfp^{(z)\intercal}\disFilter = 1$, for the iteration $(z)$\footnote{Note that when $\bfp^{(z)} = 1$, the constraint returns to the $\ell_1$ penalty.}. In particular, the method is initialized using the solution of \cite{Lin2007} and iterated enforcing sparsity using the solution of the previous problem, that is $\bfp^{(z)} = \disFilterHat^{(z-1)}$.

