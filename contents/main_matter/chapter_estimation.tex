\chapter{Acoustic Echo Retrieval}\label{chap:estimation}
\openepigraph{Signal, a function that conveys information about a phenomenon.
$[\dots]$ Consider an acoustic wave, which can convey acoustic or music information.}{R. Priemer, \textit{Introductory Signal Processing}}
\vspace{-2.5em}
\newthought{Synopsis}  blabla

\section{Problem Formulation}
The \AERdef/ problem consists in estimating the echoes' timings (or delays) $\kbrace{\tau_r}$ and attenuations (or gains) $\kbrace{\alpha_r}$.
In this context the echoes are considered the early reflections component of the \RIR/, modeled as
\begin{equation}
    \echoModelFreq
    .
\end{equation}
The term \AER/ is not an established name for such problem and depending on the field of research and the prior knowledge available, it can be referred to with different name.
In fact \AER/ can be seen as general case of Time of Arrivals Estimation or a instance of Acoustic Channel Estimation and Shaping, and Spike Retrieval and Onset Detection.
As opposed to \AER/, the task of Time of Arrivals Estimation, is only focused in estimating the echos' timings $\kbrace{\tau_r}$.
The only knowledge $\kbrace{\tau_r}$ is sufficient for typical application of related to Sound Source Localization and Room Geometry Reconstruction.
Moreover knowing $\kbrace{\tau_r}$, the attenuations $\kbrace{\alpha_r}$ can be estimated in a subsequent phase using closed-from equations~\citeonly{condat2015cadzow}.
Time of Arrivals Estimation is sometimes called Time Delays Estimation, when the origin of time is taken \wrt/ the first \TOA/ and not when sound emission.
\AER/ may be confused with Time Delay Estimation.
\AER/ may be confused with Acoustic Echo Cancellation which refers to the problem of estimating and removing...

\section{Taxonomy on \AER/ methods}

In general, we can identify four main categories which differ on whether the transmitted signal is know or not and on whether the estimation of the \RIR/.

\dichotomy{Active vs. passive approaches}.
When the input/transmitted source signal is known, then the scenario is said \textit{active}.
This type of approaches uses one or more loudspeaker to probe the environment and one or more microphones to record the propagated probe sound.
This type of methods falls into the big categories of \textit{deconvolution problem} since a ``clean'' reference signal is used to deconvolve the observed one.
Two are the main advantages of these approaches. First, this methods can be used in single-microphone scenario.
Second, with proper probe signal, a good estimation of the \RIR/ can be achieved.
Instead, passive or blind approaches are source agnostic.
To decouple environment from source signal, they leverage either on prior knowledge on the source signal or by comparing the signals received at two (or more) spatially-separated microphones.

\dichotomy{RIR-agnostic vs. RIR-based approches}.
As opposed to \textit{direct}, \textit{2-step} methods estimate the echoes' properties after estimating the (full or partial) \RIR/(s).
By modeling the early part of the \RIR/ as collection of Dirac, solving the \AER/ problem can be seen as solving two subsequent tasks:
\RIR/ estimation followed by echoes extraction.
The former can be seen as an instance of Channel Estimation problems, while the latter as a spike retrieval, pick picking or onset detection.

The latter is more challenging since, in general settings, theoretical ambiguities make this problem unsolvable~\citeonly{xu1995least, subramaniam1996cepstrum, tong1998multichannel}.
\\Instead of estimating the full acoustic impulse response, other methods estimate it partially using assumptions derived by the application.
It is the case of Impulse Response Shaping or Shortening.
In context of room acoustics, they aim to reduce the late reverberations allowing some early reflections which are perceptually useful~\citeonly{betlehem2012efficient}.
\\Direct methods instead try to surpass the challenging task of estimating the full \RIR/ and tuning peak-picking methods, and set-up an end-to-end framework.
May extract other information.

\newthought{Channel Equalization or Inverse Filtering}, is a particular case of channel estimation.
Here the goal is to estimate an equalization filter, which is the inverse of the channel impulse response, rather than the estimation of the channel impulse response itself.
This equalization filter is then applied to the received signal in order to extract the transmitted one\sidenote{\textit{Beamforming} or \textit{Spatial Filtering} can be seen as in this sense}.
In general estimating the equalization filter can be easier or more difficult than estimate the direct filter~\sidenote{If the channel impulse response is assumed to be minimum phase, the problem becomes trivial.}.
\citeonly{neely1979invertibility}, shows that even if the \RIR/ is perfectly known, its direct inversion is not straightforward.
Several techniques were investigated to alleviate an exhaustive review of Room Response Equalization can be found in~\citeonly{cecchi2018room}.

\newthought{Channel Shaping} is another approach similar to the one of channel equalization \citeonly{krishnan2015robust, vairetti2017scalable}.

Given the above definition, we can now review so \AER/ methods presented in the literature.

\section{Literature Review}

\subsection{Active and RIR-based method}
In this categories falls all the methods that first attempt for a ``good'' estimation of \RIRs/ for which the transmitted signal is known.
The echoes are identified and peaks in the estimated \RIRs/ and ad-hoc techniques are used to identify them.

\newthought{The \RIR/ Estimation step} is typically modeled as a deconvolution problem whose performances depends on the type of transmitted signal.
When the transmitted signal is arbitrary, several methods were developed to measure real \RIRs/.
Since the \RIR/ identifies the room response to a perfect impulse, one can measured by producing an impulse sound, \eg/ finger snap, a clap, piercing a ballon, or a gun shot.
Even though this methods are still used in acoustic measurement, they show clear limitation in term of reproducibly and safety.
Moreover a perfect impulse is difficult to approximate.
Instead, modern computational technique are used, involving speaker and microphone and computing the deconvolution (or cross-correlation) between an known emitted signal and the recorded output.
\\The \MLSdef/ technique was first proposed by Schroeder~\citeonly{schroeder1979integrated} and it is based on the excitation of the acoustical space by a periodic pseudo-random signal, called \MLS/.
The RIR is then calculated by circular cross-correlation between the measured output and the original \MLS/ signal.
This method was further improved in order to achieve better \RIR/ estimation in~\citeonly{dunn1993distortion,aoshima1981computer}.
Unfortunately this technique introduces several distortion artifacts which yield to spurious peaks and components in the estimation and it is sensible to the harmonic distortion introduced by the playback device, \eg/ the loudspeakers.
\\To overcome this issue, the \ESSdef/ technique introduced by Farina~\citeonly{farina2000simultaneous,farina2007advancements}.
The probe signal is the \ESS/ signal, \aka/ \textit{chirp signal}, which benefit of the following properties:
the signal span a certain frequency range; it is self-orthogonal, namely it compress into Dirac's impulse during autocorrelation; and its Fourier inverse is known in closed form.
The last property allows the user not to record and invert the probe signal.
The reader can find a review of the presented techniques in~\citeonly{szoke2019building}
\sidenote{This technique has many other propreties. On can refer to the Farina's tutorial~\cite{farina2007advancements} and this Python tutorial} %\href{https://medium.com/@baranov.mv/how-i-was-listening-to-a-sine-sweep-all-day-long-294c374995f6}}.

\mynewline
Sometimes the transmitted signal is known, but one of the above technique cannot be used.
In this scenario, \RIR/ estimation problem needs to be addressed as a more general deconvolution problem, typically solved through optimization methods~\citeonly{lin2006bayesian}.
This approaches are well studied and can be solved using standard Linear Least Squares with closed-form solution.
However, in case of narrowband signal (\eg/ speech or music) or low SNR, these problems become ill-conditioned and prior knowledge about the \RIR/ bay improve the estimation~\citeonly{boooh}.

\newthoughtpar{Echo Retrieval from RIR}
As discussed through~\cref{pt:background}, acoustic echoes can be identified as peaks in the early part of the \RIR/.
In general, such peak are not necessarily positive, thus to better visualize the \textit{echogram}\citeonly{kuttruff2016room}, namely $\rir = \magnitudeOf{\rir}$, or the energy envelop~\citeonly{schroeder1979integrated} are used instead%
\sidenote{The energy envelope typically computed as the magnitude of the analytic signal computed with the Hilbert transform.}.
Provided a good estimation of it, this peaks location and amplitudes could be extracted manually by experts.
However, even in ideal scenario, the automation of this process and the correct identification such quantities are not straightforward tasks.
As showed in \citeonly{tukuljac2018mulan}, since the arrival time of each reflection is not necessarily multiple of the sampling grid, their true location (and amplitudes) are blurred by spurious peaks.
Moreover the harmonic distortion due to the non-ideal source-receiver coupling can creates spurious spikes as well.
This issue which is referred to as \textit{basis mismatch} in the \textit{compressed sensing} community reveals the strong limitation of simple \RIR/ peak-pinking.
Although it can be alleviated by increasing the sampling frequency, it is bound to occurs in practice.
And small errors of echoes timing estimation yield to significant differences in echo-based application~\cite{defrance2008finding}.
\\The existing methods can be dichotomized into three broad categories: on-gird and off-grid approaches.
The methods belonging to the former group are the most used in practice but advance technique are used to coupe with the above mentioned issue.
~\citeonly{kuster2008reliability, crocco2017uncalibrated, remaggi2016acoustic, defrance2008detecting, bello2005tutorial, cheng2016attack, defrance2008detecting, annibale2012geometric, kelly2014detecting, usher2010improved}.
The most straightforward way is to deploy iterative and adaptive thresholding algorithm on the followed by robust and manually tuned peak finders ~\citeonly{kuster2008reliability, crocco2017uncalibrated}.
In the work~\citeonly{remaggi2016acoustic}, based on a algorithm presented in\citeonly{naylor2006estimation}, peaks are clustered according to changes in the phase slope.
Similar techniques are used in the field of music onset detection, where edge-detection wavelet filters~\citeonly{bello2005tutorial} or the attach-decay pattern~\citeonly{cheng2016attack} are used to detect event onset.
Other methods consist in peaks are selected considering the \RIR/ Kurtosis In~\citeonly{usher2010improved}.
\\By noticing that the reflection in the \RIRs/ exhibit similar shape of the direct path, the author of~\citeonly{defrance2008detecting} first proposed the use of Matching Pursuit to identified such shapes.
Here the direct sound part was used as an atom in a matching pursuit algorithm.
Unfortunately, in its pure form, matching pursuit is unsuitable for \RIRs/ because of the non-stationary nature of the reflection affected by frequency dependent characteristic of the room absorption material.
In order to improve the detection, \citeonly{kelly2014detecting} extents this approach employing Dynamic Time Warping to account for the non-uniform compression, dilation and concurrency of the early reflection.
Nevertheless, the idea of exploiting the direct path component to isolate the source-receiver coupling and thus identified first prominent reflection through deconvolution was used in~\citeonly{annibale2012geometric}.
This is known also as matching filter or direct-path compensation.
\\Alternatives approaches, detect the echo timings in other signal representation.
In \citeonly{vesa2010segmentation}  the echoes are localized in the time frequency domain using the cross-wavelet transform based on the early work~\citeonly{guillemain1996characterization, loutridis2005decomposition}.
Curiously the recent works~\citeonly{ristic2013detection,pavlovic2016multifractal} uses (multi-)fractal analysis to detect echoes in the time-frequency domain.

\mynewline
All the above mentioned works aims at detecting echoes on the sampling grid.
In order to coupe with the pathological issues of this approach, off-grid framework can be used, \eg/~\citeonly{condat2013robust}.
This approach can be related to other classical \ML/ estimation problem, which consist in selecting the model which is most likely to explain the observed noisy data.
In this categories fall other classical spectral estimation technique, \eg/ \MUSIC/~\citeonly{loutridis2005decomposition}, \ESPRIT/~\citeonly{roy1986esprit}, which are fast but statistically suboptimal.
The method presented in \citeonly{condat2013robust} focuses on the general problem of estimating a finite stream of Dirac pulses from uniform, noisy, lowpass-filtered samples.
This problem can be reformulated as a matrix denoising problem, from which the echoes location and amplitudes can be retrieved in closed-from.
Although this method reach statistical optimality in \ML/ sense, the exact knowledge of number of Diracs needs to be known in advance.
If this number is unknown or approximated, huge errors in the estimation are observed.
This consist in a huge drawbacks since a priori the exact number of echoes is difficult to know and false-positive spikes are present even in clear \RIRs/.

\mynewline
That have being said, \AER/ is far from trivial and solved even on clean \RIR/ estimate.
It is crucial to note that, for every TOA estimator, a practical trade off exists between the number of missed TOAs and the number of spurious TOAs wrongly selected.
This trade-off is only partially dependent by the SNR since, also for clean received signals, many factors can provide spurious peaks.
For instance, side lobes due to finite signal bandwidth, echo distortions due to frequency dependent attenuations and coalescing peaks due to close TOAs can affect peak estimation.
This fact is often a source of unavoidable outliers that make the robustness of subsequent steps in room estimation a delicate and very important issue.
A way to overcome this issues is to overestimate the echoes in the \RIR/ by including some false-positive observation and further prune them.

\newthought{Peak Disambiguation} to be intregrated with croccos.
\textit{Echo labeling} or \textit{\TOA/ Disambiguation} is the task of assign acoustic echoes to different image source or reflectors.
Many methods are been proposed in the context of \SSL/~\citeonly{scheuing2006disambiguation, zannini2010improved}, microphone calibration~\cite{parhizkar2014single,salvati2016sound}
and \RooGE/~\citeonly{antonacci2010geometric, filos2011robust, venkateswaran2012localizing, antonacci2012inference, dokmanic2013acoustic, crocco2014towards, jager2016room, el2017time}.
A brief review of these methods is provided in~\citeonly{crocco2017uncalibrated}.
\\In the context of \SSL/, the disambiguation is typically performed in the \TDOAs/ space~\citeonly{scheuing2006disambiguation, zannini2010improved} which are makes them more prone to error since higher precision is required\sidenote{%
For instance, at $\SI{48}{kHz}$ a \TDOA/ from the same wavefront between a sensor pair $\SI{40}{cm}$ apart and a source $\SI{2.6}{m}$ away, takes a very small value of $4.46$ samples}.
Moreover this works focuses on actively localizing multiple sources while discarding reflection, rather than localizing the image source.
The other schemes disambiguation \TOAs/ directly and are typically used in \RooGE/.
In \citeonly{venkateswaran2012localizing} the pruning of the combinatorial candidate-source search is done through Bayesian inference.
A similar approach can be found in~\citeonly{dokmanic2013acoustic, parhizkar2014single} where the validity check is based on structured matrix called Euclidean Distance Matrix and further improved using compatibility graphs in~\citeonly{jager2016room}.
These methods rely on a combinatorial search with potentially high number of candidates, which leads to intractable computational complexity.
Moreover these methods require that all the distance between each microphone needs to known with precision, which may be available in practice.
\\In the works~\citeonly{antonacci2010geometric, filos2011robust, antonacci2012inference} where reflectors are modelled as planes tangent to the ellipsoids with foci given by each pair of microphone/source.
In general these methods requires a very specific acquisition setup and use non-linear optimization in order to provide a robust solution.
All the above methods do not have specific strategies to cope with missing or spurious estimate of echoes given by malfunctioning of the peak finder or by selection of peaks corresponding to high order reflections
and is some specific case manual annotation is used.
\\Another interest approach is used in the work by~\citeonly{el2017time} which exploits the geometrical properties of linear array of loudspeakers.
By stacking side-by-side the estimated \RIRs/ in an image, the wavefront of each reflection can be easily identified and labeled.
This approach avoid the combinatorial search but still require specific setup.
\\In the work~\citeonly{crocco2014towards} an iterative strategy is used. First the direct path arrivals are used to estimate a first guess of microphone and source positions.
Then the whole set of extracted peaks and used to estimate the planar reflectors positions which are then used refine the microphone and source localization.
Iterating between geometrical space where microphone and source coordinates are defined and the signal space where the \TOAs/ are defined, the ambiguous peaks are discarded during the optimization.

\mynewline
In general this methods rely on the peak estimation pre-processing. In case of missing data, outliers


\subsection{Active and RIR-agnostic method}
This class of methods uses the signal at the microphones to directly estimate the echoes reflection thus avoiding the rir estimation and peak detection steps.
Here two approach can be identified: \ML/-based approaches~\citeonly{jensen2019method, saqib2020estimation} and auto-correlation-based approaches~\citeonly{crocco2014towards, al2019early}, which are related to the audio cameras~\citeonly{o2008imaging, o2010automatic}.
The former approach exploits the strong relation between the \TOA/ of a echo with its \DOA/.
When multiple microphone are used and their relative position is known, the relation between the two quantities can be express in closed-form.
The \DOA/ can be used to reduces the ambiguity of the estimated echoes.
This method extends a class of existing methods used in multipath communication system, denoted as \JADE/~\citeonly{vanderveen1997joint,verhaevert2004direction}.
\\Alternatively, the echoes contribution can be extracted by using the correlation function.
The correlation analysis is a mathematical tool for the identification of repeated patterns, such as the presence of a periodic or repeated signal, as function of a certain time lag.
The received signal at the microphones consists in repeated copies of the emitted signal, which is known in case of active scenario.
Therefore, the received signal correlates perfectly with the emitted one and and peaks in the correlation function can be observed.
By the extraction of these peaks, echoes timing and amplitudes can be identified.
This approach is used in \citeonly{crocco2014towards, al2019early}.
\\When the array geometry is known, the time lag axis of the correlation function can be mapped to possible angle of arrivals, namely from \TOAs/ to \DOAs/.
Therefore the identification of strong reflection can be conducted the so-called angular spectrum domain, which consider sound energy as function possible angle of arrival.
With proper clustering approach, the reflections can be identified and disambiguated.
In \citeonly{o2008imaging, o2010automatic}, after linear transform from polar to cartesian coordinates, the angular spectrum is superimposed to a panoramic picture of the audio scene taken by the barycenter of the recording arrays.
This is way this approach is called acoustic camera.
For detailed review of autocorrelation methods \citeonly{chen2006time}.
\\The main drawbacks of this method is  ...

\subsection{Passive and RIR-based method}
FIR BSI SIMO

The root in blind channel identification can be found is~\citeonly{sato1975method}
Since then, many algorithms have been proposed
Since it was recognized that the problem can be solved in the light of only SOS[2], the focus ofthe blind channel identification research has shifted to SOS methods, behind which, the motiva- tion is the potential for fast convergence. There is a rich literature on SOS blind channel identification. Many batch methods have had good success to some extent, such as the subspace (SS) algorithm [3], the cross relation (CR) algorithm [4], [5], the least squares component normalization (LSCN) algorithm [6], the linear prediction based subspace (LP-SS) algorithm [7], and the two-step maximum likelihood (TSML) algorithm
see ~\citeonly{tong1998multichannel} for a review.
\begin{itemize}
    \item subspace method: The key idea is that the channel (or part of the channel) vector is in a one- dimensional subspace of either the observation statistics or a block of noiseless observations.
    These methods, which are often referred to as subspace algorithms, have the attractive property that the channel estimates can often be obtained in a closed form from optimizing a quadratic cost function
    Pro: closed-from identification.
    Cons: as they rely on the property that the channel lies in a unique direction (subspace), they may not be robust against modeling errors.
    especially when the channel matrix is close to being singular.
    The second disadvantage is that they are often more computationally expensive.
    \item maximum likelihood method y, unlike subspace based approaches, the ML methods usually cannot be obtained in closed form.
\end{itemize}

root papers \citeonly{tukuljac2018mulan}
here multiple passive sensors.
To the best of the authors’ knowledge, all existing methods in blind channel identification rely on the discrete-time model
Blind Channel Estimation~\citeonly{xu1995least, tong1998multichannel}
\begin{itemize}
    \item Subspace methods
    \item Cross Relation methods
    \item Sparsity based cross-relation
    \item Maximum likelihood
\end{itemize}

BCE for RIR estimation. root paper \citeonly{crocco2016estimation}
\citeonly{tong1994blind,lin2007blind,lin2008blind,kowalczyk2013blind,crocco2015room,crocco2016estimation}
\begin{itemize}
    \item introduction of the cross relation~\citeonly{tong1994blind}. SIMO can be formulated as a eigenvalue problem - least square minimization
    \item anchor constrain and non negative~\citeonly{lin2007blind}
    \item bayesian learning for lambda~\citeonly{lin2008blind}
    \item spit Bergam methods~\citeonly{kowalczyk2013blind}. which is known to be robust to back- ground noise.
    \item iterative weighted l1 constrain~\citeonly{crocco2015room} and improved in~\citeonly{crocco2016estimation}.
    to better balance sparsity penalty and sparsity constraints.
    \item Kalman \citeonly{qi2019broadband}
\end{itemize}

BSI for RIR, root paper ~\citeonly{qi2019broadband}
\begin{itemize}
    \item Subspace mathods
    \item Linear Prediction Coding
    \item Cross Relation
    \item RTF with oversampling~\citeonly{koldovsky2015sparse}
\end{itemize}

BCE for RIR estimation, root paper \citeonly{kowalczyk2013blind}
\begin{itemize}
    \item SIMO-BSI which exploits spatial differences between multiple microphone signals, has received significant attention as it does not require any assumption about the source signal and can be formulated as an adaptive algorithm~\citeonly{huang2003class}
    \item in \citeonly{chen2006time} to outperform crosscorrelation-based techniques, such as generalized crosscorrelation (GCC), for source localization in noisy and reverberant environments. At the cost of complexity.
\end{itemize}

Under this perspective, \RTF/ estimation techniques can be used.
As mentioned in~\cref{subsec:processing:rtf}, the \RTF/ can be
RTF estimation in the special case $h_i = 1$
In particular scenarios, when microphone is close to one source, the RIR at another microphone RTF estimation methods can be used~\citeonly{koldovsky2015spatial}.

limitation of this methods:

% Although they rely on sparsity-enforcing regularizers, the filters are strictly-speaking non- sparse in practice, due to the sinc kernel (Fig. 1). This general bottle-neck in compressed sensing has been referred to as basis mismatch and was notably studied in [28]. In particular, the true peaks of the filters do not correspond to the true echoes (Fig. 1), even for N to inf. Though, most existing methods rely on peak-picking [18, 5].
\begin{itemize}
    \item For the same reason, these methods are fundamentally on-grid, namely, they can only output echo locations which are integer multiple of the sampling period 1/Fs. This pre- vents subsample resolution, which may be important in applications such as room shape reconstruction from audio signals [3].
    \item These methods strongly rely on the knowledge of the length L of the filters. However, due to the sinc kernel (Sec. 2.1.1), the true filters are always infinite.
    \item The dimension of the search space isML-1, which is much larger in practice than the actual number 2MK of unknown variables. This makes the methods computationally demanding and sometimes intractable for large filter lengths (typically in the tens of thousands for acoustic
\end{itemize}

After it, peak picking

\subsection{Passive and RIR-agnostic method}
Note that the real channel is not perfectly non-negative since the Dirac pulses, corresponding to the walls reflections, typically fall off the grid of samples resulting in sinc functions. Despite this slight mismatch between theoretical assumptions and real data, the position of the estimated peaks reproduces the positions of the ground truth peaks with remarkable precision.
This are also off-grid methods
\begin{itemize}
    \item MULAN \citeonly{tukuljac2018mulan}
    \item Blaster \citeonly{di2020blaster}
    \item Audio cameras - beamforming~\citeonly{sun2011joint}
\end{itemize}

% Approaches
% \begin{itemize}
%     \item assuming ideal propagation
%     \begin{description}
%         \item[M=2, Cross-correlation approaches]
%         \item[M=2, Generalized Cross-correlation approaches]
%         \item[M=2, LMS-type adaptive]
%         \item[Sensor Fusion] SRP-PHAT
%     \end{description}
%     \item assuming reverberant path
%     \begin{description}
%         \item[Adaptive Eigenvalue decomposition]
%     \end{description}
% \end{itemize}

% Is the transmitted signal known?
% \begin{description}
%     \item[Yes] Active Channel Estimation
%     \begin{itemize}
%         \item Deconvolution followed by Peak Picking methods
%         \begin{description}
%             \item[Deconvolution]
%             \begin{itemize}
%                 \item Regression and Deconvolution
%                 \item Informed Deconvolution (Sparsity)
%                 \item RTF estimation in the special case $h_i = 1$
%                 \item RIR estimation \citeonly{Farina}
%                 \item cross correlation
%             \end{itemize}
%             \item[Peak Picking]
%             \begin{itemize}
%                 \item Iterative Thresholding (Crocco)
%                 \item Condat - Sinc Denoising
%                 \item Defrance - Matching Pursuit
%                 \item Remmagi, Naylor
%                 \item Onset detection
%                 \item Kurtosis
%                 \item Autocorrelation
%                 \item Matched filter / Direct Path compensation
%                 \item Geometry-based
%                 \item DTW - Kelly
%                 \item Multifractals analysis
%                 \item Time-Frequency representation (Wavelet) - Vesa and Lokki
%             \end{itemize}
%         \end{description}
%         \item Jesper Jensen
%         \begin{itemize}
%             \item EM and DOA
%             \item \cite{crocco2014towards}
%             \item Active Acoustic Cameras
%             \item autocorrelation \citeonly{al2019early}
%         \end{itemize}
%     \end{itemize}
%     \item[No]. Do we have prior Knowledge?
%     \begin{description}
%         \item[Yes] Statistics methods
%             \item 2nd order
%             \item ML
%             \item leglaive
%         \item[No] Is it on grid? Does it estimate the full channel? Require peak picking?
%             \begin{description}
%                 \item[Yes] On grid Blind Method
%                     \begin{itemize}
%                     \item subspace method
%                     \item cross-relation methods
%                     \end{itemize}
%                 \item[No]
%                 \begin{itemize}
%                     \item MULAN
%                     \item Blaster
%                     \item Audio camera
%                 \end{itemize}
%             \end{description}
%     \end{description}
% \end{description}

\section{Data and Evaluation}

\subsection{Datasets}
Syntethic RIRs

Real RIRs

\subsection{Metrics}
The metrics used in \AER/ depend on the application and the methods used to estimate the echoes.
It following that two types of metrics can be identified.

\newthoughtpar{BCE metrics}
For application related to channel estimation, \eg/ speech enhancement, typically both echoes timing and amplitudes are required.

\newthoughtpar{Information Retrival Metrics}
aoeu

% \section{as (sparse) \RIR/ estimation}
% \begin{itemize}
%     \item[def] estimation of the whole channel acoustic channel
%     \item[methods] Signal know \vs/ unknown. statistical methods \vs/ blind method
% \end{itemize}

% \subsection{Other echo-related parameters estimation}
% \newthought{\TDOA/ estimation}
% \begin{itemize}
%     \item[def] estimation of the the difference of the direct path
%     \item[methods] Cross-correlation
% \end{itemize}

% \newthought{Echo Density Estimation}

% \newthought{$\RT$ and $\DRR$ estimation}

% \section{Acoustic Echo Estimation is}

% \begin{itemize}
%     \item Acoustic Echo Retrieval definition
%     \item Acoustic Echo Retrieval scope and placement in the signal processing pipeline
%     \item Acoustic Echo Retrieval characteristic
% \end{itemize}

% \section{Echoes in the Time, Frequency and Cepstral domains}
% \begin{itemize}
%     \item Time domain processing
%     \item Frequency domain processing
%     \item Correlation processing
%     \item Cepstral processing
% \end{itemize}


% \section{Related Works}
% \subsection{Active vs. Passive echoes estimation}

% \subsection{Knowledge-based vs. Data-driven}
% \begin{itemize}
%     \item Knowledge-driven (Physic-driven)
%     \begin{itemize}
%         \item Channel (RIR) estimation and Echoes pruning - Crooco and Dokmanic
%         \item TDOA estimation (multipath) - Benesty
%         \item Spikes Retrieval - Condat
%     \end{itemize}
%     \item Data-driven
%     \begin{itemize}
%         \item GLLiM
%         \item Deep Leaning echo estimation
%     \end{itemize}
% \end{itemize}

% \subsection{end-2-end vs 2-steps approaches}

% AER\\
% eRTF + AER

% Pruning methods

% \section{Related Works}

% \subsection{AER as a RIR Estimation problem}
% \todo{summarize Crocco's presentation}

% TX signal: known vs. not known
% \\TX signal not known: statistical methods and blind methods

% \subsection{AER as a Spike Estimation problem}


% \subsection{Virtually-supervised and Data Augmentation}


% \section{Data and Metrics}
% \subsection{Spike-based metrics}