\chapter{Acoustic Echo Retrieval}\label{chap:estimation}
\openepigraph{Signal, a function that conveys information about a phenomenon.
$[\dots]$ Consider an acoustic wave, which can convey acoustic or music information.}{R. Priemer, \textit{Introductory Signal Processing}}
\vspace{-2.5em}
\newthought{Synopsis}  blabla

\section{Problem Formulation}
The \AERdef/ problem consists in estimating the echoes' timings (or delays) $\kbrace{\tau_r}$ and attenuations (or gains) $\kbrace{\alpha_r}$.
In this context the echoes are considered the early reflections component of the \RIR/, modeled as
\begin{equation}
    \echoModelFreq
    .
\end{equation}
The term \AER/ is not an established name for such problem and depending on the field of research and the prior knowledge available, it can be referred to with different name.
In fact \AER/ can be seen as general case of Time of Arrivals Estimation or a instance of Acoustic Channel Estimation and Shaping, and Spike Retrieval and Onset Detection.
As opposed to \AER/, the task of Time of Arrivals Estimation, is only focused in estimating the echos' timings $\kbrace{\tau_r}$.
The only knowledge $\kbrace{\tau_r}$ is sufficient for typical application of related to Sound Source Localization and Room Geometry Reconstruction.
Moreover knowing $\kbrace{\tau_r}$, the attenuations $\kbrace{\alpha_r}$ can be estimated in a subsequent phase using closed-from equations~\citeonly{condat2015cadzow}.
Time of Arrivals Estimation is sometimes called Time Delays Estimation, when the origin of time is taken \wrt/ the first \TOA/ and not when sound emission.
\AER/ may be confused with Time Delay Estimation.
\AER/ may be confused with Acoustic Echo Cancellation which refers to the problem of estimating and removing...

\section{Taxonomy on of Acoustic Echo Retrieval methods}

In general, we can identify four main categories which differ on whether the transmitted signal is know or not and on whether the estimation of the \RIR/.

\dichotomy{Active vs. passive approaches}.
When the input/transmitted source signal is known, then the scenario is said \textit{active}.
This type of approaches uses one or more loudspeaker to probe the environment and one or more microphones to record the propagated probe sound.
This type of methods falls into the big categories of \textit{deconvolution problem} since a ``clean'' reference signal is used to deconvolve the observed one.
Two are the main advantages of these approaches. First, this methods can be used in single-microphone scenario.
Second, with proper probe signal, a good estimation of the \RIR/ can be achieved.
Instead, passive or blind approaches are source agnostic.
To decouple environment from source signal, they leverage either on prior knowledge on the source signal or by comparing the signals received at two (or more) spatially-separated microphones.

\dichotomy{RIR-agnostic vs. RIR-based approches}.
As opposed to \textit{direct}, \textit{2-step} methods estimate the echoes' properties after estimating the (full or partial) \RIR/(s).
By modeling the early part of the \RIR/ as collection of Dirac, solving the \AER/ problem can be seen as solving two subsequent tasks:
\RIR/ estimation followed by echoes extraction.
The former can be seen as an instance of Channel Estimation problems, while the latter as a spike retrieval, pick picking or onset detection.

The latter is more challenging since, in general settings, theoretical ambiguities make this problem unsolvable~\citeonly{xu1995least, subramaniam1996cepstrum, tong1998multichannel}.
\\Instead of estimating the full acoustic impulse response, other methods estimate it partially using assumptions derived by the application.
It is the case of Impulse Response Shaping or Shortening.
In context of room acoustics, they aim to reduce the late reverberations allowing some early reflections which are perceptually useful~\citeonly{betlehem2012efficient}.
\\Direct methods instead try to surpass the challenging task of estimating the full \RIR/ and tuning peak-picking methods, and set-up an end-to-end framework.
May extract other information.

\newthought{Channel Equalization or Inverse Filtering}, is a particular case of channel estimation.
Here the goal is to estimate an equalization filter, which is the inverse of the channel impulse response, rather than the estimation of the channel impulse response itself.
This equalization filter is then applied to the received signal in order to extract the transmitted one\sidenote{\textit{Beamforming} or \textit{Spatial Filtering} can be seen as in this sense}.
In general estimating the equalization filter can be easier or more difficult than estimate the direct filter~\sidenote{If the channel impulse response is assumed to be minimum phase, the problem becomes trivial.}.
\citeonly{neely1979invertibility}, shows that even if the \RIR/ is perfectly known, its direct inversion is not straightforward.
Several techniques were investigated to alleviate an exhaustive review of Room Response Equalization can be found in~\citeonly{cecchi2018room}.

\newthought{Channel Shaping} is another approach similar to the one of channel equalization \citeonly{krishnan2015robust, vairetti2017scalable}.

Given the above definition, we can now review so \AER/ methods presented in the literature.

\section{Literature Review}

\subsection{Active and RIR-based method}
In this categories falls all the methods that first attempt for a ``good'' estimation of \RIRs/ for which the transmitted signal is known.
The echoes are identified and peaks in the estimated \RIRs/ and ad-hoc techniques are used to identify them.

\newthought{The \RIR/ Estimation step} is typically modeled as a deconvolution problem whose performances depends on the type of transmitted signal.
When the transmitted signal is arbitrary, several methods were developed to measure real \RIRs/.
Since the \RIR/ identifies the room response to a perfect impulse, one can measured by producing an impulse sound, \eg/ finger snap, a clap, piercing a ballon, or a gun shot.
Even though this methods are still used in acoustic measurement, they show clear limitation in term of reproducibly and safety.
Moreover a perfect impulse is difficult to approximate.
Instead, modern computational technique are used, involving speaker and microphone and computing the deconvolution (or cross-correlation) between an known emitted signal and the recorded output.
\\The \MLSdef/ technique was first proposed by Schroeder~\citeonly{schroeder1979integrated} and it is based on the excitation of the acoustical space by a periodic pseudo-random signal, called \MLS/.
The RIR is then calculated by circular cross-correlation between the measured output and the original \MLS/ signal.
This method was further improved in order to achieve better \RIR/ estimation in~\citeonly{dunn1993distortion,aoshima1981computer}.
Unfortunately this technique introduces several distortion artifacts which yield to spurious peaks and components in the estimation and it is sensible to the harmonic distortion introduced by the playback device, \eg/ the loudspeakers.
\\To overcome this issue, the \ESSdef/ technique introduced by Farina~\citeonly{farina2000simultaneous,farina2007advancements}.
The probe signal is the \ESS/ signal, \aka/ \textit{chirp signal}, which benefit of the following properties:
the signal span a certain frequency range; it is self-orthogonal, namely it compress into Dirac's impulse during autocorrelation; and its Fourier inverse is known in closed form.
The last property allows the user not to record and invert the probe signal.
The reader can find a review of the presented techniques in~\citeonly{szoke2019building}
\sidenote{This technique has many other propreties. On can refer to the Farina's tutorial~\citeonly{farina2007advancements} and this Python tutorial} %\href{https://medium.com/@baranov.mv/how-i-was-listening-to-a-sine-sweep-all-day-long-294c374995f6}}.

\mynewline
Sometimes the transmitted signal is known, but one of the above technique cannot be used.
In this scenario, \RIR/ estimation problem needs to be addressed as a more general deconvolution problem, typically solved through optimization methods~\citeonly{lin2006bayesian}.
This approaches are well studied and can be solved using standard Linear Least Squares with closed-form solution.
However, in case of narrowband signal (\eg/ speech or music) or low SNR, these problems become ill-conditioned and prior knowledge about the \RIR/ bay improve the estimation~\citeonly{boooh}.

\newthoughtpar{Echo Retrieval from RIR}
As discussed through~\cref{pt:background}, acoustic echoes can be identified as peaks in the early part of the \RIR/.
In general, such peak are not necessarily positive, thus to better visualize the \textit{echogram}\citeonly{kuttruff2016room}, namely $\rir = \magnitudeOf{\rir}$, or the energy envelop~\citeonly{schroeder1979integrated} are used instead%
\sidenote{The energy envelope typically computed as the magnitude of the analytic signal computed with the Hilbert transform.}.
Provided a good estimation of it, this peaks location and amplitudes could be extracted manually by experts.
However, even in ideal scenario, the automation of this process and the correct identification such quantities are not straightforward tasks.
As showed in \citeonly{tukuljac2018mulan}, since the arrival time of each reflection is not necessarily multiple of the sampling grid, their true location (and amplitudes) are blurred by spurious peaks.
Moreover the harmonic distortion due to the non-ideal source-receiver coupling can creates spurious spikes as well.
This issue which is referred to as \textit{basis mismatch} in the \textit{compressed sensing} community reveals the strong limitation of simple \RIR/ peak-pinking.
Although it can be alleviated by increasing the sampling frequency, it is bound to occurs in practice.
And small errors of echoes timing estimation yield to significant differences in echo-based application~\citeonly{defrance2008finding}.
\\The existing methods can be dichotomized into three broad categories: on-gird and off-grid approaches.
The methods belonging to the former group are the most used in practice but advance technique are used to coupe with the above mentioned issue.
~\citeonly{kuster2008reliability, crocco2017uncalibrated, remaggi2016acoustic, defrance2008detecting, bello2005tutorial, cheng2016attack, defrance2008detecting, annibale2012geometric, kelly2014detecting, usher2010improved}.
The most straightforward way is to deploy iterative and adaptive thresholding algorithm on the followed by robust and manually tuned peak finders ~\citeonly{kuster2008reliability, crocco2017uncalibrated}.
In the work~\citeonly{remaggi2016acoustic}, based on a algorithm presented in\citeonly{naylor2006estimation}, peaks are clustered according to changes in the phase slope.
Similar techniques are used in the field of music onset detection, where edge-detection wavelet filters~\citeonly{bello2005tutorial} or the attach-decay pattern~\citeonly{cheng2016attack} are used to detect event onset.
Other methods consist in peaks are selected considering the \RIR/ Kurtosis In~\citeonly{usher2010improved}.
\\By noticing that the reflection in the \RIRs/ exhibit similar shape of the direct path, the author of~\citeonly{defrance2008detecting} first proposed the use of Matching Pursuit to identified such shapes.
Here the direct sound part was used as an atom in a matching pursuit algorithm.
Unfortunately, in its pure form, matching pursuit is unsuitable for \RIRs/ because of the non-stationary nature of the reflection affected by frequency dependent characteristic of the room absorption material.
In order to improve the detection, \citeonly{kelly2014detecting} extents this approach employing Dynamic Time Warping to account for the non-uniform compression, dilation and concurrency of the early reflection.
Nevertheless, the idea of exploiting the direct path component to isolate the source-receiver coupling and thus identified first prominent reflection through deconvolution was used in~\citeonly{annibale2012geometric}.
This is known also as matching filter or direct-path compensation.
\\Alternatives approaches, detect the echo timings in other signal representation.
In \citeonly{vesa2010segmentation}  the echoes are localized in the time frequency domain using the cross-wavelet transform based on the early work~\citeonly{guillemain1996characterization, loutridis2005decomposition}.
Curiously the recent works~\citeonly{ristic2013detection,pavlovic2016multifractal} uses (multi-)fractal analysis to detect echoes in the time-frequency domain.
Alternatively two recent works operates in the cepstral domain~\citeonly{ferguson2019improved,jia2017extraction}.
The \textit{cepstrum} is the spectrum of a logarithmic spectrum and is used in practice to detect periodicity in the spectral domain.
This approach seems promising since time-domain spikes are mapped as complex sinusoids in frequency and used in the past for source-filter deconvolution.
However this representation is highly sensible to noise affecting the observation and the accuracy is limited by the approximation of the \DFT/ operator.

\mynewline
All the above mentioned works aims at detecting echoes on the sampling grid.
In order to coupe with the pathological issues of this approach, off-grid framework can be used, \eg/~\citeonly{condat2013robust}.
This approach can be related to other classical \ML/ estimation problem, which consist in selecting the model which is most likely to explain the observed noisy data.
In this categories fall other classical spectral estimation technique, \eg/ \MUSIC/~\citeonly{loutridis2005decomposition}, \ESPRIT/~\citeonly{roy1986esprit}, which are fast but statistically suboptimal.
The method presented in \citeonly{condat2013robust} focuses on the general problem of estimating a finite stream of Dirac pulses from uniform, noisy, lowpass-filtered samples.
This problem can be reformulated as a matrix denoising problem, from which the echoes location and amplitudes can be retrieved in closed-from.
Although this method reach statistical optimality in \ML/ sense, the exact knowledge of number of Diracs needs to be known in advance.
If this number is unknown or approximated, huge errors in the estimation are observed.
This consist in a huge drawbacks since a priori the exact number of echoes is difficult to know and false-positive spikes are present even in clear \RIRs/.

\mynewline
That have being said, \AER/ is far from trivial and solved even on clean \RIR/ estimate.
It is crucial to note that, for every TOA estimator, a practical trade off exists between the number of missed TOAs and the number of spurious TOAs wrongly selected.
This trade-off is only partially dependent by the SNR since, also for clean received signals, many factors can provide spurious peaks.
For instance, side lobes due to finite signal bandwidth, echo distortions due to frequency dependent attenuations and coalescing peaks due to close TOAs can affect peak estimation.
This fact is often a source of unavoidable outliers that make the robustness of subsequent steps in room estimation a delicate and very important issue.
A way to overcome this issues is to overestimate the echoes in the \RIR/ by including some false-positive observation and further prune them.

\newthought{Peak Disambiguation} to be intregrated with croccos.
\textit{Echo labeling} or \textit{\TOA/ Disambiguation} is the task of assign acoustic echoes to different image source or reflectors.
Many methods are been proposed in the context of \SSL/~\citeonly{scheuing2006disambiguation, zannini2010improved}, microphone calibration~\citeonly{parhizkar2014single,salvati2016sound}
and \RooGE/~\citeonly{antonacci2010geometric, filos2011robust, venkateswaran2012localizing, antonacci2012inference, dokmanic2013acoustic, crocco2014towards, jager2016room, el2017time}.
A brief review of these methods is provided in~\citeonly{crocco2017uncalibrated}.
\\In the context of \SSL/, the disambiguation is typically performed in the \TDOAs/ space~\citeonly{scheuing2006disambiguation, zannini2010improved} which are makes them more prone to error since higher precision is required\sidenote{%
For instance, at $\SI{48}{kHz}$ a \TDOA/ from the same wavefront between a sensor pair $\SI{40}{cm}$ apart and a source $\SI{2.6}{m}$ away, takes a very small value of $4.46$ samples}.
Moreover this works focuses on actively localizing multiple sources while discarding reflection, rather than localizing the image source.
The other schemes disambiguation \TOAs/ directly and are typically used in \RooGE/.
In \citeonly{venkateswaran2012localizing} the pruning of the combinatorial candidate-source search is done through Bayesian inference.
A similar approach can be found in~\citeonly{dokmanic2013acoustic, parhizkar2014single} where the validity check is based on structured matrix called Euclidean Distance Matrix and further improved using compatibility graphs in~\citeonly{jager2016room}.
These methods rely on a combinatorial search with potentially high number of candidates, which leads to intractable computational complexity.
Moreover these methods require that all the distance between each microphone needs to known with precision, which may be available in practice.
\\In the works~\citeonly{antonacci2010geometric, filos2011robust, antonacci2012inference} where reflectors are modelled as planes tangent to the ellipsoids with foci given by each pair of microphone/source.
In general these methods requires a very specific acquisition setup and use non-linear optimization in order to provide a robust solution.
All the above methods do not have specific strategies to cope with missing or spurious estimate of echoes given by malfunctioning of the peak finder or by selection of peaks corresponding to high order reflections
and is some specific case manual annotation is used.
\\Another interest approach is used in the work by~\citeonly{el2017time} which exploits the geometrical properties of linear array of loudspeakers.
By stacking side-by-side the estimated \RIRs/ in an image, the wavefront of each reflection can be easily identified and labeled.
This approach avoid the combinatorial search but still require specific setup.
\\In the work~\citeonly{crocco2014towards} an iterative strategy is used. First the direct path arrivals are used to estimate a first guess of microphone and source positions.
Then the whole set of extracted peaks and used to estimate the planar reflectors positions which are then used refine the microphone and source localization.
Iterating between geometrical space where microphone and source coordinates are defined and the signal space where the \TOAs/ are defined, the ambiguous peaks are discarded during the optimization.

\mynewline
In general this methods rely on the peak estimation pre-processing. In case of missing data, outliers


\subsection{Active and RIR-agnostic method}
This class of methods uses the signal at the microphones to directly estimate the echoes reflection thus avoiding the rir estimation and peak detection steps.
Here two approach can be identified: \ML/-based approaches~\citeonly{jensen2019method, saqib2020estimation} and auto-correlation-based approaches~\citeonly{crocco2014towards, al2019early}, which are related to the audio cameras~\citeonly{o2008imaging, o2010automatic} and \SSL/~\citeonly{dibiase2001robust}.
The former approach exploits the strong relation between the \TOA/ of a echo with its \DOA/.
When multiple microphone are used and their relative position is known, the relation between the two quantities can be express in closed-form.
The \DOA/ can be used to reduces the ambiguity of the estimated echoes.
This method extends a class of existing methods used in multipath communication system, denoted as \JADE/~\citeonly{vanderveen1997joint,verhaevert2004direction}.
\\Alternatively, the echoes contribution can be extracted by using the correlation function.
The correlation analysis is a mathematical tool for the identification of repeated patterns, such as the presence of a periodic or repeated signal, as function of a certain time lag.
The received signal at the microphones consists in repeated copies of the emitted signal, which is known in case of active scenario.
Therefore, the received signal correlates perfectly with the emitted one and and peaks in the correlation function can be observed.
By the extraction of these peaks, echoes timing and amplitudes can be identified.
This approach is used in \citeonly{crocco2014towards, al2019early}.
\\When the array geometry is known, the time lag axis of the correlation function can be mapped to possible angle of arrivals, namely from \TOAs/ to \DOAs/.
Therefore the identification of strong reflection can be conducted the so-called angular spectrum domain, which consider sound energy as function possible angle of arrival.
With proper clustering approach, the reflections can be identified and disambiguated.
In \citeonly{o2008imaging, o2010automatic}, after linear transform from polar to cartesian coordinates, the angular spectrum is superimposed to a panoramic picture of the audio scene taken by the barycenter of the recording arrays.
This is way this approach is called acoustic camera.
For detailed review of autocorrelation methods \citeonly{chen2006time}.
cross-correlation among acquired signals are poorly performing, especially with noisy environments, transmitted signals of limited bandwidth and low energy reflections with respect to the direct path\citeonly{chen2006time}.
The main drawbacks...
% in \citeonly{chen2006time} to outperform crosscorrelation-based techniques, such as generalized crosscorrelation (GCC), for source localization in noisy and reverberant environments. At the cost of complexity.

\subsection{Passive and RIR-based method}
The passive approach relies on using external sound sources in the environment to conduct the estimation.
One obvious advantage of such approaches is that they are non-intrusive since only already existing sounds are used in the estimation.
In the literature this problem belongs to the broad and deeply studied category of \BCE/ or \BSI/.
Since such explicit reference signal is available, and the delay estimate is often acquired by comparing the signals received at two (or more) spatially separated sensors.
This approach results is \SIMO/ \BSI/ and is a long-standing and still active research topic in signal processing, notably due to its fundamental ill-posedness.
In the context of \RIR/ estimation, a number of methods have been developed~\citeonly{}.
Common to all this method is the assumption that \RIR/ are discrete \FIR/ filters define on the sampling grid, namely a vector.
In the general setting of arbitrary signals and filters, rigorous theoretical ambiguities under which the problem is unsolvable have been identified~\citeonly{xu1995least}.
Some well-known limitations of these approaches are their sensitivity to the chosen length of filters, and their intractability when the filters are too large.
\\The innovative idea of passive \BCE/ can be traced back to~\citeonly{sato1975method}~\sidenote{a review of the the evolution of \SIMO/ \BCE/ can be found in~\citeonly{huang2003class}}.
Since then, many algorithms have been proposed.
They can be broadly dichotomized into the class of statistical method and the class of (completely) blind method.

\mynewline
Since the nature of the source signal is by definition not deterministic, their statistic can known based on the signal category, \eg/ speech or music.
Based on this, statistical approaches exploits such knowledge and, as described in ~\citeonly{tong1998multichannel}, two main approaches can be identified:
\begin{description}
    \item[The second order moments] approaches derives closed-form solution for which the knowledge of the source autocorrelation function is required.
    \item[Maximum Likelihood] requires instead source probability function.
    Even though the are optimal in \ML/ sense, they does not produce closed-form solution, but rather non-convex cost function which are solved via \EM/.
    In this categories one may include the methods developed for informed multichannel blind source separation~\citeonly{ozerov2009multichannel,duong2010under,leglaive2016multichannel,leglaive2018student,scheibler2018separake}.
    Here the source signals are typically modeled as Gaussian distribution centred in zero and unknown variances.
    Using pre-trained dictionaries for modeling the variances of the source, the are able to estimated both the acoustic channels and the source contribution.
    In particular the work~\citeonly{duong2010under} extends this framework for reverberant recordings using physic-bases models for the late reverberations, while~\citeonly{leglaive2016multichannel} consider explicitly the contribution of early echoes, further improved in~\citeonly{leglaive2018student}.
\end{description}
However, in general, statistical methods play a minor role in \RIR/ estimation due to the difficulty in achieving reliable statistics of the emitted signal or good initialization point required by the \EM/.
While for the source separation approaches, even if the the final estimated \RIRs/ may match the real one in the statistical sense, they lacks of accuracy, indispensable for \AER/.

\newthought{Blind methods comprises two main groups:} \textit{subspace} methods~\citeonly{abed1997subspace} and \textit{cross-relation methods}~\citeonly{tong1994blind,xu1995least,lin2007blind,lin2008blind,kowalczyk2013blind,crocco2015room,crocco2016estimation}.

The former is based on the key idea that the channel (or part of it ) vector spans a one-dimensional subspace of (block of of) noiseless observations.
This methods have the attractive property that the channel estimates can often be obtained in a closed form from optimizing a quadratic cost function
However they relay on the they may not be robust, especially when the channel covariance matrix is close to being singular.
The second disadvantage is that they are typically computationally expensive.
\\The second family of methods rely on the clever observation that in noiseless case, for every pair of microphone $(i, i')$, it holds
\begin{equation}
    (c_{i'} \convDis h_i)[n] = (c_{i} \convDis h_{i'})[n] =  ((h_{i'} \convDis h_i) \convDis s)[n]
    ,
\end{equation}
by the commutativity of the convolution operator.
This principle is called the \textit{cross-relation} and it was firstly introduced by~\citeonly{tong1994blind}.
In this work, the \RIR/ are estimated by solving a Least Square minimization of the square cross relation errors.
In \citeonly{xu1995least,tong1998multichannel} sufficient and necessary condition for channel identification are discussed.
This approach has received significant attention as it does not require any assumption about the source signal.
Later, the accuracy of estimated \RIRs/ has been subsequently improved using a priori knowledge of the filters:
in particular, the authors of~\citeonly{lin2007blind} have proposed to use sparsity penalty and non-negativity constraints to increase robustness to noise as well as Bayesian-learning methods to automatically infer the value of the hyper-parameters in~\citeonly{lin2008blind}.
Even if sparsity and non-negativity could be seen as a strong assumption, works in speech enhancement~\citeonly{ribeiro2010turning,dokmanic2015raking} and room geometry~\citeonly{antonacci2012inference,crocco2017uncalibrated} estimation have proven the effectiveness of this approach.
On a similar scheme, in~\citeonly{kowalczyk2013blind},~\eqref{eq:xrel_toepl} is solved using an adaptive time-frequency-domain approach while~\citeonly{aissa2008blind} proposes to use the $\ell_p$-norm instead of the $\ell_1$-norm.
A successful approach has been presented recently by Crocco \textit{et al.} in \citeonly{crocco2015room,crocco2016estimation}, where the anchor constraint is replaced by an \textit{iterative weighted} $\ell_1$ equality constraint.
to better balance sparsity penalty and model constraints.
These approaches will be further detailed in~\cref{ch:blaster}.
Finally the recent work~\citeonly{qi2019broadband} extends this approach under the umbrella of the Kalman filter which was previously used for echo-cancellation application.

\mynewline
An alternative approach is used in~\citeonly{vcmejla2019mirage}, where the \RIR/ estimation problem is treated as special case of \RTF/ estimation problem.
As mentioned in~\cref{subsec:processing:rtf}, the \RTF/ can be estimated when the reference microphone is placed very close to the source.
\RTF/ estimation found its root in the field of speech enhancement~\citeonly{gannot2001signal} and many techniques have been proposed since then~\citeonly{gannot2001signal,koldovsky2015spatial,koldovsky2015sparse,kodrasi2017evd}.
Methods for \RTF/ estimation will be detailed in~\cref{ch:brioche}.
In general, by its definition, \RTF/ describes the relative filter between two observations and not directly their \RIRs/.
Therefore the extraction of the echo properties is far from trivial and in some cases impossible.
The main limitation of this approach is that it is possible only in measurement scenario, where the user have the possibility to place microphone arbitrarily in the room.
Nevertheless, in this perspective, this assumption is found to be useful not only for \RTF/ estimation, but also for microphone calibration, since
some geometrical ambiguities can be easily solved in closed-form, as done in~\citeonly{crocco2012closed}.

\mynewline
The main limitation of all recent \FIR/ \SIMO/ \BCE/ works is that they rely on sparsity-enforcing regularizers and peak-picking.
As described in~\cref{sec:processing:fouriermodel}, due to the measurement process involving a l$\sinc$ function, the filters are strictly speaking non-sparse and non-negative in practice.
This general bottle-neck in compressed sensing has been referred to as \textit{basis mismatch} and was notably studied in~\citeonly{chi2011sensitivity}.
In particular, the true peaks in the \RIR/ do not necessarily correspond to the true echoes.
Since these methods are fundamentally on-grid, the estimated echo locations integer multiple of the sampling period $1/\Fs$.
This prevents subsample resolution, which may be important in applications such as \RooGE/~\citeonly{crocco2017uncalibrated} or acoustic parameter estimation~\citeonly{defrance2008finding}.
Moreover, these methods strongly rely on the knowledge of the length of the filters.
When this parameter is underestimated or overestimated, identifiability and computational issue may arise, affecting the estimation.
Nevertheless, despite this slight mismatch between theoretical assumptions and real data, for some scenario the position of the estimated peaks by the methods~\citeonly{crocco2016estimation} reproduces the positions of the ground truth peaks with remarkable precision as demonstrated our work~\citeonly{di2020blaster}.

\subsection{Passive and RIR-agnostic method}
Methods in this categories pass the onerous task of estimating the (full or partial) acoustic channel and, to the best author knowledge, are only a few have been identified.
As for the active and RIR-agnostic case, the audio cameras based on the cross-correlation function can be used.
Exploiting the geometrical knowledge of the microphone array, \TDOAs/ extracted from robust correlation function can be mapped to \DOAs/~\citeonly{dibiase2001robust,o2008imaging,o2010automatic}.
Assuming a single source scenario, difference \DOAs/ can be disambiguated using geometrical prior knowledge and can be associated to image sources, hence reflectors.
This methods typically ignore the echoes amplitudes and in general do consider only angles on the unit sphere, ignoring the distance from the source.
Without proper prior knowledge, their application to \AER/ is far from trivial as \RooGE/ and Reflector estimation method needs to be used to convert \DOAs/ them back to echoes timing.
\\Recently a fully blind, passive, off-grid and RIR-agnostic method was proposed by~\citeauthor{tukuljac2018mulan} for stereophonic recordings, namely using only 2 microphones.
They proposed a method, called \MULAN/, based on the properties of the \textit{annihilation filter}, the work of~\citeonly{condat2013robust} and the theory of \FRI/, attempt to operate directly in the parameter space of the echo location and amplitudes, by
For a sequence of Fourier coefficients (describing a signal or a filter), its annihilation filter is such that the linear convolution between the sequence and the filter coefficients is identically zero.
If the source signal is known, starting from the cross-relation identity, the \AER/ problem translates in finding the annihilation filter for the \RIRs/, which can be recasted into a eigenvalue problem.
In the fully blind case, the problem is solved with non-convex optimization, iterating between the estimation of the two filters and the signal until convergence.
The method was later extended to multichannel case in~\citeonly{peic2020sparse} using the generalization of Cadzow denoising framework~\citeonly{condat2015cadzow}.
This method is show to outperform conventional methods by several orders of magnitude in precision in noiseless case, with synthetic data and when the correct number of echoes is known a priori.
However its effectiveness is not being tested on challenging real scenario featuring external noise and partial knowledge on the number of echo.


\section{Data and Evaluation}
\AER/ is relatively recent problem which is typically addressed in the context of much broader application, \eg/ \SE/, \RooGE/, \SSL/.
Therefore the literature lacks of standard datasets as well as standard evaluation framework.

\subsection{Datasets}
As listed in \citeonly{Szoke2018building} and in \citeonly{Genovese2019blind}, a number of recorded RIRs corpora are available online and for free, each of them meeting the demands of certain applications, usually \SE/ and \ASR/.
However, even if these datasets feature reverberation and strong early reflections, they lack proper annotations, making them difficult to use for testing \AER/ methods.
For this reason, to bypass the complexity of recording real annotated RIR datasets, simulators based on the \ISM/ are extensively used instead%
~\citeonly{}.
While simulated datasets are more versatile, simple and quicker to obtain, they fail to fully capture the complexity and the richness of real acoustic environments.
Due to this, methods trained or validated on them may fail to generalize to real conditions, as will be shown in~\cref{ch:dechorate}.

\mynewline
A good dataset for \AER/ should include a variety of environments (rooms geometries and surface materials), of microphone placings (close to or away from reflectors, scattered or forming ad-hoc arrays) and, most importantly, precise annotations of the scene's geometry and echo timings within the \RIRs/.
Moreover, in order to be versatile and used in echo-aware applications, the provided annotations should match the \ISM/, \textit{i.e.}, TOAs should be expressed in terms of image sources and vice-versa.
Such data are difficult to collect since they require precise measurements of the positions and orientations of all the acoustic emitters, receivers and reflective surfaces inside the environment with dedicated planimetric equipment.
We identified two main classes of related RIR datasets in the literature:
\SE//\ASR/-oriented datasets, e.g. \citeonly{Szoke2018building, Bertin2019voice, Cmejla2019mirage}, and \RooGE/-oriented datasets, e.g. \citeonly{Dokmanic2013acoustic, Crocco2017uncalibrated, remaggi2017acoustic}.
The formers include acoustic echoes as highly correlated interfering source coming from close reflectors, such as desk in meeting rooms or the close wall, however their proper annotations are not provided.
The latter group deals with sets of distributed, synchronized microphones and loudspeakers in a room.
These setups are not exactly suitable for \SE/ methods, which typically involve compact or ad hoc arrays. Table~\ref{tab:rir_db} summarizes some existing datasets.

% Please add the following required packages to your document preamble:
% \usepackage{multirow}
\begin{table*}[h!]
\label{tab:rir_db}
\small
\centering

\begin{tabular}{l|c|c|c|ccc|c|l|l}
\toprule

\multirow{2}{*}{Database Name} &
  \multicolumn{3}{c|}{Annotated} &
  \multicolumn{4}{c|}{Number of} &
  Key characteristics &
  Purpose \\
  &
  Pos. &
  Echoes &
  Rooms &
  \multicolumn{1}{l|}{RIRs} &
  \multicolumn{1}{c|}{Rooms} &
  \multicolumn{1}{c|}{Mic$\times$Pos.} &
  Src &
   &
   \\
\hline
\begin{tabular}[c]{@{}l@{}} Dokmani\'c \textit{et al.} \citeonly{Dokmanic2013acoustic}\end{tabular} &
  \cmark  &
  $\sim$ &
  \multicolumn{1}{c|}{$\sim$} &
  \multicolumn{1}{l|}{15} &
  \multicolumn{1}{c|}{3} &
  \multicolumn{1}{c|}{5} &
  1 &
  Non shoebox room &
  RooGE \\ \hline
\begin{tabular}[c]{@{}l@{}} Crocco \textit{et al.} \citeonly{crocco2017uncalibrated}\end{tabular} &
  \cmark &
  $\sim$ &
  \multicolumn{1}{c|}{\cmark} &
  \multicolumn{1}{l|}{204} &
  \multicolumn{1}{c|}{1} &
  \multicolumn{1}{c|}{17} &
  12 &
  \begin{tabular}[c]{@{}l@{}}Accurate 3D calibration\\ Many mic and src positions\end{tabular} &
  RooGE \\
 \hline

 Remaggi \textit{et al.} \citeonly{remaggi2016acoustic} &
  \cmark & % # noete pos?
  $\sim$  & % # note echo?
  \cmark & % # note room?
  \multicolumn{1}{l|}{$\sim$1.5k} & % # rirs
  \multicolumn{1}{c|}{4} & % # room
  \multicolumn{1}{c|}{48$\times$2} & % # mics
  4-24 & % # srcs
  \begin{tabular}[c]{@{}l@{}}
  Circural dense array
  \\Circular placement of sources
  \end{tabular} &
  \begin{tabular}[c]{@{}l@{}}RooGE\\ SE$^{\dagger}$\end{tabular}
  \\
\hdashline
Remaggi \textit{et al.} \citeonly{Remaggi2019modeling} &
  \cmark & % # noete pos?
  $\sim$  & % # note echo?
  \cmark & % # note room?
  \multicolumn{1}{l|}{$\sim$1.6k} & % # rirs
  \multicolumn{1}{c|}{4} & % # room
  \multicolumn{1}{c|}{
    \begin{tabular}[c]{@{}l@{}}
    48$\times$2
    \\+2$\times$2
    \end{tabular}
} & % # mics
  3-24 & % # srcs
  \begin{tabular}[c]{@{}l@{}}
    Circural dense array
    \\Binaural Recordings
    \end{tabular}
  &
  \begin{tabular}[c]{@{}l@{}}RooGE$^{\dagger}$\\ SE\end{tabular}
  \\
\hline
BUT Reverb dB \citeonly{Szoke2018building} &
  \cmark &
  \xmark&
  $\sim$ &
  \multicolumn{1}{l|}{$\sim$1.3k} &
  \multicolumn{1}{c|}{8} &
  \multicolumn{1}{c|}{(2-10)$\times$6} &
  3-11 &
  \begin{tabular}[c]{@{}l@{}}Accurate metadata\\ different device/arrays\\ various rooms\end{tabular} &
  SE/ASR \\ \hline
VoiceHome \citeonly{Bertin2019voice} &
   \cmark &
   \xmark&
   \xmark&
  \multicolumn{1}{l|}{188} &
  \multicolumn{1}{c|}{12} &
  \multicolumn{1}{c|}{8$\times$2} &
  7-9  &
   Various rooms, real homes &
  SE/ASR \\ \hline
dEchorate &
  \textbf{\cmark} &
  \textbf{\cmark} &
  \multicolumn{1}{c|}{\textbf{\cmark}} &
  \multicolumn{1}{l|}{$\sim$1.8k} &
  \multicolumn{1}{c|}{1} &
  \multicolumn{1}{c|}{30} &
  6 &
  \begin{tabular}[c]{@{}l@{}}Accurate annotation\\ Different Echo-energy\end{tabular} &
  \begin{tabular}[c]{@{}l@{}}RooGE\\ SE/ASR\end{tabular}
  \\

\bottomrule
\end{tabular}
\caption{Comparison between some existing RIR databases that account for early acoustic reflections. Receiver positions are indicated in terms of number of microphones per array times number of different positions of the array ($\sim$ stands for partially available information).
The read is invited to refer to \citeonly{Szoke2018building, Genovese2019blind} for more complete list of existing RIR datasets.
%\protect\\- indicates minimum and maximum number of variable number of objects
\protect\\$^{\dagger}$The dataset in \citeonly{remaggi2016acoustic} is originally intended for RooGE and further extended for (binaural) SE in \citeonly{Remaggi2019modeling} with a similar setup.}
\end{table*}

\subsection{Metrics}
Again, the metrics used in \AER/ depend on the application and the methods used to estimate the echoes.
The following types of metrics can be identified:

\newthoughtpar{\FIR/ \SIMO/ \BCE/ metrics}
As described above \AER/ can be seen as instance of \FIR/ \SIMO/ \BCE/ problems.
The related method models the ground-truth acoustic channels as a discrete vector $\hath \in \bbR^L$.
Similarly their estimate is in the same way, namely, $\hath^{*}[n] \in \bbR^L$.
To assert the quality of the estimated discrete filters the following metrics have been proposed in the literature:
\begin{description}
    \item[The \RMSEtxt/] is a well know metrics that measure the mean distance between two vector as points in the Euclidean space.
    \begin{equation}
        \RMSE(\hath^{*}, \hath) = \sqrt{\sum_{n=0}^{L-1} \kparen{\hath^{*}[n] - \hath[n]}^2}
        .
    \end{equation}
    This metrics is known to be highly sensible to the scaling, and little mismatch between a few echoes.
    For instance, if the $\hath^{*}$ is a just a shifted and scaled version of the $\hath$
    \item[projection misalignment] By projecting onto and defining a projection error, we take into account only the intrinsic misalignment of the channel estimate, disregarding an arbitrary gain facto
    \item[Cross-ralation error]
    \item[Blocking ability]
    \citeonly{huang2003class}

    normalized projection misalignment~\citeonly{morgan1998evaluation, ahmad2006proportionate}
\end{description}

\newthoughtpar{\RooGE/ metrics}
\begin{description}
    \item[Position \RMSEtxt/]
    \item[]
\end{description}

\newthoughtpar{Information Retrival Metrics}
\begin{description}
    \item[\TOAs/ \RMSEtxt/]
    \item[Precision, Recall, and F-measure]
\end{description}