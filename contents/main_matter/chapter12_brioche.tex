\chapter{\brioche: Speech Enhancement with Echoes}\label{ch:brioche}

\marginpar{%
\footnotesize
Spatial Filtering, Acoustic Reflection, Relative Transfer Function, Beamforming, Room Impulse Response
}
\newthought{Synopsis}

\section{Introduction}

\subsection{Literature review: an acoustic perspective}
Bibliography with respect to sound propagation

\subsection{Literature review an algorithmic perspective}
Bibliography with respect to learning and knowledge approaches

\newthoughtpar{Rake-receivers}

\newthoughtpar{RTF-based beamformes}

\section{Background in SE}

\newthoughtpar{Type of beamformes}

\section{\brioche: Beamforming with Echoes}

\section{Experimental Evaluation}

\section{Metrics}

\href{https://github.com/aliutkus/speechmetrics}

\newthoughtpar{PESQ}
\begin{itemize}
    \item presented in~\citeonly{rix2001perceptual}
    \item a measure assumed to cover several speech degradation and distortion was promoted in the ITU-T recommendation P.862 [ITU-T 2001]
    \item This metric named “Perceptual Evaluation of Speech Quality” (PESQ) is for now considered as one of the most reliable metric to predict the overall speech quality
    \item PESQ was first designed to account mainly for network or telecommunication distortion
    \item the signals are equalized following the typical frequency response of a telephone
    \item Practically, after applying an auditory model to the signals (based on a Bark frequency scale) the loudness spectra are estimated.
    \item From the loudness spectra differences (disturbance), the wideband PESQ predicts a Mean Opinion Score (MOS) as it could be retrieved from genuine listening tests.
    \item Mean opinion score (MOS) is a measure used in the domain of Quality of Experience and telecommunications engineering, representing overall quality of a stimulus or system. It is the arithmetic mean over all individual "values on a predefined scale that a subject assigns to his opinion of the performance of a system quality".
    \item The MOS is expressed as a single rational number, typically in the range 1 to 5
    \item PESQ output scores range from 1 (bad) to 5 (excellent).
\end{itemize}

\newthoughtpar{SRMR}
\begin{itemize}
    \item proposed in~\citeonly{falk2010non}
    \item Cochlear implant preprocessing: The SRMR is obtained by first applying a 23-channel gammatone filterbank to the time domain signal.
    \item A Hilbert transform captures the envelope of the output of each filter, and thus the temporal dynamics information.
    \item These signals are then transformed into the STFT domain to obtain an 8 band modulation spectrum segmented into frames.
    \item The SRMR is then obtained by comparing the energy in the different bands of the modulation spectrum
    \item Lastly, the output value is computed as the ratio of the average modulation energy content available in the first four modulation bands (from 3 Hz up to 15 Hz (20 Hz) for CI, thus consistent with clean speech modulation content [16])
         to the average modulation energy content available in the higher frequency modulation bands (circa 20–160 Hz for the SRMR, 15–82 Hz for the CI-adapted version).
\end{itemize}




\section{Conclusion}
